\begin{flushright}
    \textbf{\Large{Kuis 1}}
    \subsection*{Tahun 2022}
    \addcontentsline{toc}{subsection}{Kuis 1 - 2022}
\end{flushright}
\vspace{0.5cm}
\hrule height 2pt
\vspace{0.5cm}
\begin{center}
    \textbf{\large{MATERI}}
    \begin{enumerate}[leftmargin=*, label={\arabic*}.]
        \item Menyelesaikan pertidaksamaan yang melibatkan bentuk rasional.
        \item Fungsi komposisi dan menentukan domainnya.
        \item Mencari nilai limit dari fungsi.
        \item Menentukan kekontinuan fungsi pada suatu titik.
    \end{enumerate}
\end{center}
\vspace{0.2cm}
\hrule height 1pt
\vspace{0.5cm}
\begin{center}
    \textbf{\large{SOAL}}
\end{center}
\begin{enumerate}[leftmargin=*, label={\arabic*}.]
\item Diketahui $f(x) = x$ dan $\ds g(x) = \frac{x}{x+4}$.
\begin{enumerate}[label={\alph*}.]
    \item Tentukanlah nilai $x$ yang memenuhi $f(x) \leq g(x)$.
    \item Jika $h(x) = \sqrt{f(x)}$, maka tentukanlah $(h \circ g)(x)$ 
    dan domain dari $(h \circ g)(x)$. Jelaskanlah jawaban anda!
\end{enumerate}
\item Misalkan diberikan fungsi berikut ini: 
$\ds k(x) = \frac{x^{2}-4}{x^{2}-5x+6}$
\begin{enumerate}[label={\alph*}.]
    \item Tentukan diskontinuitas dari fungsi $k$ dan jelaskan mengapa fungsi
    itu gagal kontinu pada titik-titik tersebut.
    \item Tentukan titik-titik yang dapat dihapus dan yang tidak dapat dihapus
    diskontinuitasnya.
\end{enumerate}
\end{enumerate}
\vspace{0.2cm}
\hrule height 1pt
\vspace{0.5cm}

\begin{center}
    \textbf{\large{PEMBAHASAN}}
\end{center}
\begin{enumerate}[leftmargin=*, label={\arabic*}.]
\item
\begin{enumerate}[label={\alph*}.]
\item Akan dicari nilai $x$ yang memenuhi
$\ds f(x) \leq g(x) \iff x \leq \frac{x}{x+4}$.

\vspace{0.1cm}
Hanya ada 3 aturan dasar yang diperlukan untuk menyelesaikan pertidaksamaan.
\begin{enumerate}[label={\arabic*})]
    \item Jumlahkan kedua ruas dengan bilangan yang sama.
    \item Kalikan kedua ruas dengan bilangan positif yang sama.
    \item Kalikan kedua ruas dengan bilangan negatif yang sama dan 
    mengubah arah pertidaksamaannya.
\end{enumerate}
Soal ini melibatkan bentuk rasional. Lakukan pembagian kasus sehingga kedua
ruas dapat dikalikan $(x+4)$
    
\textbf{Kasus 1: }$x+4 > 0$ atau $x>-4$\\
Maka $x+4$ adalah bilangan \textbf{positif} dan
\begin{align*}
    x \leq \frac{x}{x+4}
    \iff &x(x+4) \leq x
    &\text{kedua ruas kalikan $(x+4)$}\\
    \iff &x^{2}+4x \leq x
    &\text{penyederhanaan}
\end{align*}
Bagi menjadi subkasus lagi sehingga kedua ruas dapat dibagi $x$\\
\textbf{Subkasus 1.1: }$x>-4$ dan $x>0$\\
Maka $x$ adalah bilangan \textbf{positif} dan saat dilanjutkan
\begin{align*}
    x^{2}+4x \leq x
    \iff &x+4 \leq 1
    &\text{kedua ruas kalikan $\frac{1}{x}$}\\
    \iff &x \leq -3
    &\text{kedua ruas jumlahkan $-4$}
\end{align*}
Didapat nilai $x$ yang memenuhi adalah $x\leq -3$. Ini berkontradiksi dengan
kasus ini yang dimana $x > 0$. Sehingga tidak ada nilai $x$ yang memenuhi
pada subkasus ini. \\
\textbf{Subkasus 1.2: }$x>-4$ dan $x<0$\\
Maka $-4 < x < 0$ dan $x$ adalah bilangan \textbf{negatif}, saat dilanjutkan
\begin{align*}
    x^{2}+4x \leq x
    \iff &x+4 \geq 1
    &\text{kedua ruas kalikan $\frac{1}{x}$}\\
    \iff &x \geq -3
    &\text{kedua ruas jumlahkan $-4$}
\end{align*}
Didapat nilai $x$ yang memenuhi adalah $x \geq -3$. Syarat pada subkasus
ini adalah $-4 < x < 0$ sehingga nilai $x$ yang memenuhi pada subkasus ini
adalah $-3 \leq x < 0$ \\
\textbf{Subkasus 1.3: }$x>-4$ dan $x=0$\\
Maka pertidaksamaan menjadi $0 \leq 0$ yang bernilai benar. Nilai $x=0$
memenuhi pertidaksamaan pada subkasus ini.

\textbf{Kasus 2: }$x+4 < 0$ atau $x < -4$\\
Maka $x+4$ dan $x$ adalah bilangan \textbf{negatif} dan
\begin{align*}
    x \leq \frac{x}{x+4}
    \iff &x(x+4) \geq x
    &\text{kedua ruas kalikan $(x+4)$}\\
    \iff &x^{2}+4x \geq x
    &\text{penyederhanaan}\\
    \iff &x+4 \leq 1
    &\text{kedua ruas kalikan $\frac{1}{x}$}\\
    \iff &x \leq -3
    &\text{kedua ruas jumlahkan $-4$}
\end{align*}
Didapat nilai $x$ yang memenuhi adalah $x \leq -3$. Syarat pada kasus
ini adalah $x < -4$ sehingga nilai $x$ yang memenuhi pada kasus ini
adalah $x < -4$ \\
    
Semua nilai $x$ yang memenuhi $f(x) \leq g(x)$ adalah gabungan nilai $x$ 
dari semua kasus. Sehingga nilai $x$ yang memenuhi adalah 
$\set*{-3 \leq x < 0 \cup x = 0 \cup x < -4}$ atau 
$\set*{x \in \mathbb{R} \mid x \geq x < -4 \cup -3 \leq x \leq 0}$
atau $\oic*{-\infty, -4} \cup \cic*{-3,0}$
    
$\therefore$ Nilai $x$ yang memenuhi $f(x) \leq g(x)$ adalah
$\set*{x \in \mathbb{R} \mid x \geq x < -4 \cup -3 \leq x \leq 0}$\\
atau $\oic*{-\infty, -4} \cup \cic*{-3,0}$
    
\vspace{0.1cm}
\textbf{Catatan:}\\
Ini merupakan cara yang menerapkan konsep dasar dari penyelesaian
pertidaksamaan. Untuk pembahasan selanjutnya akan digunakan cara yang 
lebih cepat.
\begin{center}
    \line(1,0){150}
\end{center}
\item Akan dicari $(h \circ g)(x)$ dan domainnya.\\
Soal memberikan informasi bahwa $f(x) = x$, $g(x) = \frac{x}{x+4}$ 
dan $h(x) = \sqrt{f(x)} = \sqrt{x}$.\\
Dengan definisi fungsi komposisi, maka
\[
    (h \circ g)(x) = h(g(x)) = h\brk*{\frac{x}{x+4}} 
    = \sqrt{\frac{x}{x+4}}
\]
Untuk menemukan domainnya, carilah irisan dari domain  
$\ds \sqrt{\frac{x}{x+4}}$ dan $g(x)=\ds\frac{x}{x+4}$\\
Pertama karena terdapat bilangan rasional, maka penyebutnya tidak boleh
nol, sehingga $x+4 \neq 0$ atau $x\neq-4$. Lalu karena ada bentuk akar, 
maka nilai didalam akarnya harus nonnegatif.
\begin{align*}
    \frac{x}{x+4} \geq 0.
\end{align*}
Maka $x=0$ dan $x+4=0$ adalah titik stasioner.

\vspace{0.2cm}
\begin{tikzpicture}
\draw[stealth-stealth] (-6,0) node[below]{$-\infty$}--(5,0) node[below]{$\infty$};
\draw (-6,1) --(-2,1);
\draw (2,1) --(5,1);
\draw (-2,1)--(-2,-.1) node[below=0.2em]{$-4$};
\draw (2,1)--(2,-.1) node[below=0.2em]{$0$};
    
\node at (3.5,.5) {$+$};
\node at (-4,.5) {$+$};
\node at (0,.5) {$-$};
\node [draw, shape = circle, fill = black, minimum size = 0.1cm, inner sep=0pt] at (2,0){};
\node [draw, shape = circle, fill = white, minimum size = 0.1cm, inner sep=0pt] at (-2,0){};
\end{tikzpicture}\\
Sehingga domainnya adalah $x < 4 \cup x \geq 0$.

Lalu domain dari $g$ adalah semua nilai $x \neq -4$
Sehingga domain dari $(h \circ g)(x)$ adalah
\[
\set*{x \in R \mid x < 4 \cup x \geq 0} \cap \set*{x \in R \mid x \neq 4} 
= \set*{x \in R \mid x < 4 \cup x \geq 0}
\]

$\therefore$ $\ds (h \circ g)(x) = \sqrt{\frac{x}{x+4}}$ dengan domain 
$\set*{x \in R \mid x < 4 \cup x \geq 0}$ 
\begin{center}
    \line(1,0){300}
\end{center}
\end{enumerate}
\item
\begin{enumerate}[label={\alph*}.]
\item Akan dicari titik dikontinuitas fungsi $k$\\
Tiga syarat untuk membuktikan $k$ kontinu di $x=c$ adalah
\begin{enumerate}[label={\arabic*}.]
    \item $\lim_{x\to c} f(x)$ ada.
    \item $f(c)$ ada.
    \item $\lim_{x\to c} f(x) = f(c)$
\end{enumerate}
$k$ adalah fungsi yang memiliki bentuk rasional. Diskontinuitas terjadi saat 
di titik dimana penyebut bernilai nol, karena fungsi $k$ tidak terdefinisi 
di titik itu dan syarat kedua kontinuitas tidak terpenuhi.
Jika titik $c$ adalah pembuat nol penyebut maka
\begin{align*}
    c^{2} - 5c + 6 = 0
    \iff &(c-2)(c-3) = 0
    &\text{faktorisasi}
\end{align*}
sehingga titik dikontinuitas $k$ adalah saat $x=2$ dan $x=3$ karena syarat kedua 
tidak terpenuhi.

$\therefore$ $k$ diskontinu di $x=2$ dan $x=3$ karena $k(x)$ tidak ada pada 
$x$ tersebut.
\begin{center}
    \line(1,0){150}
\end{center}
\item Hasil dari soal sebelumnya akan diuji. Diskontinuitas dapat dihapus jika limit 
fungsi pada titik tersebut ada. Sehingga pendefinisian ulang fungsi dapat membuat 
syarat kedua dan ketiga kontinuitas terpenuhi.

Akan dicari limit $k$ di titik saat $x=2$ dan $x=3$.\\
Untuk $x=2$
\begin{align*}
    \lim_{x \to 2} \frac{x^{2}-4}{x^{2}-5x+6}
    &= \lim_{x \to 2} \frac{(x-2)(x+2)}{(x-2)(x-3)}
    &\text{faktorisasi}\\
    &= \lim_{x \to 2} \frac{x+2}{x-3}
    &\text{kalikan $\frac{1/(x-2)}{1/(x-2)}$ (karena $x \neq 2$)}\\
    &= \frac{2+2}{2-3} = -4
    &\text{teorema subtitusi}
\end{align*}
Karena nilai limitnya ada maka dikontinuitas di $x=2$ dapat dihapus.\\
Untuk $x=3$\\
Limitny tidak ada, karena saat $x$ mendekati $3$ maka $x-2$ mendekati $1$, 
$x+2$ mendekati $5$ dan $x-3$ mendekati 0. Sehingga limit dari fungsi ini 
saat $x$ mendekati $3$ adalah bentuk yang mendekati $\frac{5}{0}$. Dari kanan 
bentuk ini mendekati $\infty$ dan dari kiri mendekati $-\infty$ sehingga limit 
kiri dan limit kanannya tidak ada.\\
Karena nilai limitnya tidak ada maka diskontinuitas di $x=3$ tidak dapat dihapus.

$\therefore$ Titik yang dikontinuitasnya dapat dihapus adalah $x=2$ dan yang tidak 
dapat dihapus adalah $x=3$.
\begin{center}
    \line(1,0){300}
\end{center}
\end{enumerate}
\end{enumerate}
