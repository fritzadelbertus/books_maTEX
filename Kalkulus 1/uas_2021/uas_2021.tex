\newpage
\begin{flushright}
    \textbf{\Large{Ujian Akhir Semester}}
    \subsection*{Tahun 2021}
    \addcontentsline{toc}{subsection}{UAS - 2021}
\end{flushright}
\vspace{0.5cm}
\hrule height 2pt
\vspace{0.5cm}
\begin{center}
    \textbf{\large{MATERI}}
    \begin{enumerate}[leftmargin=*, label={\arabic*}.]
        \item Teorema Dasar Kalkulus
        \item Aplikasi integral dalam mencari volume benda putar
        \item Fungsi Invers
        \item Fungsi Trigonometri dan Invers Trigonometri
        \item Teknik-teknik Integrasi
    \end{enumerate}
\end{center}
\vspace{0.2cm}
\hrule height 1pt
\vspace{0.5cm}
\begin{center}
    \textbf{\large{SOAL}}
\end{center}
\begin{enumerate}[leftmargin=*, label={\arabic*}.]
\item Dengan menggunakan Teorema Dasar Kalkulus, tentukanlah $F'(x)$ jika diketahui:
\[
F(x) = \int_{-x^{2}}^{5x}\frac{s^{2}}{1+s^{2}}\,ds
\]
\item Sketsa dan carilah volume benda putar yang terbentuk, jika daerah yang 
diwarnai biru pada gambar di bawah diputar mengelilingi sumbu-$y$ dengan 
menggunakan:
\begin{enumerate}[label={\alph*}.]
    \item Metode Cincin
    \item Metode Kulit Tabung
\end{enumerate}
\begin{center}
    \begin{tikzpicture}[>=stealth]
    \begin{axis}[
        xmin=-0.2,xmax=1.2,
        ymin=-0.2,ymax=1,
        axis x line=middle,
        axis y line=middle,
        axis line style=<->,
        xlabel={$x$},
        ylabel={$y$},
        ]
        \addplot [name path=f, no marks,blue, domain=0:0.25*pi,samples=50]({tan(deg(x))},{x});
        \node [blue] at (0.5,0.7){$x=\tan y$};
        \addplot [name path=g, no marks,red, domain=0:0.25*pi,samples=50]({1},{x});
        \addplot [thick, color=cyan, fill=cyan, fill opacity=0.5]
        fill between[of=f and g, soft clip={domain=0:1.1},];
    \end{axis}
    \end{tikzpicture}
\end{center}

\item Misalkan $\ds f(x)=\frac{a^{x}-1}{a^{x}+1}$ untuk $a$ tetap, $a>0$, $a\neq 1$.
\begin{enumerate}[label={\alph*}.]
    \item Jelaskan mengapa $f$ mempunyai invers.
    \item Carilah $f^{-1}$.
\end{enumerate}
\item Carilah $\ds \frac{dy}{dx}$ di titik $\ds P\brk*{\frac{1}{2},-\sqrt{2}}$ jika diketahui:
\[
y\arccos(xy) = \frac{-3\sqrt{2}}{4}\pi
\]
\item Hitunglah integral berikut:
\begin{enumerate}[label={\alph*}.]
    \item $\ds \int\frac{2x^{3}+x^{2}+4}{\brk*{x^{2}+4}^{2}}\,dx$
    \item $\ds \int x^{3}e^{-3x}\,dx$
\end{enumerate}
\end{enumerate}
\vspace{0.2cm}
\hrule height 1pt
\vspace{0.5cm}
\begin{center}
    \textbf{\large{PEMBAHASAN}}
\end{center}
\begin{enumerate}[leftmargin=*, label={\arabic*}.]
\item Halo
\end{enumerate}

\begin{center}
    \line(1,0){300}
\end{center}