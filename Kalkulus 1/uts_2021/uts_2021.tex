\newpage
\begin{flushright}
    \textbf{\Large{Ujian Tengah Semester}}
    \subsection*{Tahun 2021}
    \addcontentsline{toc}{subsection}{UTS - 2021}
\end{flushright}
\vspace{0.5cm}
\hrule height 2pt
\vspace{0.5cm}
\begin{center}
    \textbf{\large{MATERI}}
    \begin{enumerate}[leftmargin=*, label={\arabic*}.]
        \item Menyelesaikan pertidaksamaan yang melibatkan nilai mutlak.
        \item Mensketsa grafik fungsi.
        \item Fungsi komposisi dan menentukan domain dan rangenya.
        \item Mencari nilai limit kiri dan nilai limit kanan fungsi.
        \item Mencari nilai limit fungsi di tak hingga.
        \item Menentukan kekontinuan fungsi pada suatu titik.
        \item Menyelesaikan masalah turunan implisit.
        \item Menyelesaikan permasalahan maksimum dan minimum.
    \end{enumerate}
\end{center}
\vspace{0.2cm}
\hrule height 1pt
\vspace{0.5cm}
\begin{center}
    \textbf{\large{SOAL}}
\end{center}
\begin{enumerate}[leftmargin=*, label={\arabic*}.]
\item Tentukan $\ds\frac{dy}{dx}$ dan $\ds\frac{d^{2}y}{dx^{2}}$ dari:
\begin{enumerate}[label={\alph*}.]
    \item $x^{3}+3x^{2}y-6xy^{2}+2y^{3}=0$ pada titik $(1,1)$.
    \item $y^{4}+\cos\bracket*{x^{2}y^{3}}+3y^{2}+x^{2}=5$ pada titik $(0,1)$.
\end{enumerate}
\item Diketahui $f(x)=\sqrt{x}$ dan $g(x)=x^{2}-3$. Tentukanlah:
\begin{enumerate}[label={\alph*}.]
    \item $h(x)$ jika diketahui $h(x)=(f \circ g)(x)$.
    \item Domain dari $h(x)$.
    \item Apakah $h(x)$ termasuk fungsi ganjil, genap atau bukan keduanya?
    \item Gambar grafik dari $h(x)$.
\end{enumerate}
\item Tangki tanpa tutup bervolume $1125$ cm$^3$, dengan alas persegi bersisi $x$ 
cm dan kedalaman $y$ cm. Permukaan atas tangki sejajar dengan tanah. Tangki akan 
dibangun untuk menampung air hujan. Biaya pembuatan tangki tidak hanya melibatkan 
material pembuatan tangki namun juga biaya penggalian tanah yang sebanding dengan 
hasil kali $xy$. Jika biaya total adalah $C=5(x^{2}+4xy)+10xy$ rupiah per cm$^2$, 
berapakah ukuran $x$ dan $y$ yang meminimumkan biaya?
\item Tentukan nilai $a$ dan $b$ agar fungsi berikut kontinu untuk setiap bilangan 
real $x$.
\[
    f(x)=
    \begin{cases}
        ax+5b, &x \leq 0,\\
        x^2+3a-b, &0 < x \leq 2,\\
        3x-5, &x > 2.
    \end{cases}
\]
\item Diketahui panjang jalan antara kota $A$ dan kota $B$ adalah $100$ km, dengan 
batas kecepatan tempuh adalah $100$ km/jam. Misalkan mobil $X$ berangkat dari kota 
$A$ menuju kota $B$ pada jam keberangkatan yang sama dengan mobil $Y$ berangkat 
dari kotab $B$ menuju kota $A$. Kedua mobil berpapasan disuatu titik setelah 
menempuh 30 menit waktu perjalanan.
\begin{enumerate}[label={\alph*}.]
    \item Tunjukkan bahwa salah satu mobil memiliki kecepatan tempuh melebihi $100$ 
    km/jam.
    \item  Andaikan selama perjalanan kecepatan tempuh mobil $X$ tidak pernah melebihi
    $90$ km/jam, berapakah kecepatan tempuh sesaat mobil $Y$ selama perjalanan?
\end{enumerate}
\end{enumerate}
\vspace{0.2cm}
\hrule height 1pt
\vspace{0.5cm}
\begin{center}
    \textbf{\large{PEMBAHASAN}}
\end{center}
\begin{enumerate}[leftmargin=*, label={\arabic*}.]
\item 
\begin{enumerate}[label={\alph*}.]
    \item Pertama akan dicari $\ds\frac{dy}{dx}$ dari 
    $x^{3}+3x^{2}y-6xy^{2}+2y^{3}=0$ pada titik $(1,1)$.

    Turunkan secara implisit:
    \begin{align*}
        &\drv{x}{x^{3}+3x^{2}y-6xy^{2}+2y^{3}}=\drv{x}{0}\\
        \iff &\drv{x}{x^{3}}+\drv{x}{3x^{2}y}+\drv{x}{-6xy^{2}}
        +\drv{x}{2y^{3}} = 0
    \end{align*}
    Selesaikan masing-masing turunan:
    \begin{enumerate}[label={\arabic*})]
    \item \[
    \drv{x}{x^{3}} = 3x^{2}
    \]
    \item \[
    \drv{x}{3x^{2}y} = \drv{x}{3x^{2}}y+3x^{2}\drv{x}{y} = 6xy+3x^{2}y'
    \]
    \item 
    \begin{align*}
        \drv{x}{-6xy^{2}} &= \drv{x}{-6x}y^{2} + (-6x)\drv{x}{y^{2}}
        &\text{aturan perkalian}\\
        &=(-6)y^{2}-6x\bracket*{2y\drv{x}{y}}
        &\text{aturan rantai}\\
        &=-6y^{2}-12xyy'
    \end{align*}
    \item Gunakan aturan rantai
    \begin{align*}
        \drv{x}{2y^{3}} = 6y^{2}\drv{x}{y} = 6y^{2}y'
    \end{align*}
    \end{enumerate}
    Diperoleh
    \begin{align*}
        &\drv{x}{x^{3}}+\drv{x}{3x^{2}y}+\drv{x}{-6xy^{2}}+\drv{x}{2y^{3}} = 0\\
        \iff &3x^{2}+(6xy+3x^{2}y')+(-6y^2-12xyy')+6y^{2}y'=0
    \end{align*}
    Subtitusi $x=1$ dan $y=1$ diperoleh
    \begin{align*}
        &3(1)^{2}+(6(1)(1)+3(1)^{2}y')+(-6(1)^{2}-12(1)(1)y')+6(1)^2y'=0\\
        \iff &3+6+3y'-6-12y'+6y'=0\\
        \iff &3-3y'=0\\
        \iff &y' = 1
    \end{align*}
    Diperoleh $\ds\frac{dy}{dx}$ dari 
    $x^{3}+3x^{2}y-6xy^{2}+2y^{3}=0$ pada titik $(1,1)$ adalah $1$.

    Kedua akan dicari $\ds\frac{d^{2}y}{dx^{2}}$ dari 
    $x^{3}+3x^{2}y-6xy^{2}+2y^{3}=0$ pada titik $(1,1)$.

    Sebelumnya telah diperoleh
    \[
        3x^{2}+(6xy+3x^{2}y')+(-6y^2-12xyy')+6y^{2}y'=0
    \]
    Turunkan lagi secara implisit
    \begin{align*}
        &\drv{x}{3x^{2}+(6xy+3x^{2}y')+(-6y^2-12xyy')+6y^{2}y'} = \drv{x}{0}\\
        \iff &\drv{x}{3x^{2}}+\drv{x}{6xy}+\drv{x}{3x^{2}y'}
        +\drv{x}{-6y^{2}}+\drv{x}{-12xyy'}+\drv{x}{6y^{2}y'}=0
    \end{align*}
    Selesaikan masing-masing turunan
    \begin{enumerate}[label={\arabic*})]
    \item \[
        \drv{x}{3x^{2}} = 6x
    \]
    \item \[
        \drv{x}{6xy} = \drv{x}{6x}y + 6x\drv{x}{y} = 6y+6xy'
    \]
    \item \[
        \drv{x}{3x^{2}y'} = \drv{x}{3x^{2}}y' + 3x^{2}\drv{x}{y'} = 6xy'+3x^{2}y''
    \]
    \item \[
        \drv{x}{-6y^{2}} = -12y\drv{x}{y} = -12yy'
    \]
    \item \begin{align*}
        \drv{x}{-12xyy'} &= \drv{x}{-12xy}y'+(-12xy)\drv{x}{y'}
        &\text{aturan perkalian}\\
        &=\drv{x}{-12xy}y'-12xyy''\\
        &=\bracket*{\drv{x}{-12x}y+(-12x)\drv{x}{y}}y'-12xyy''
        &\text{aturan perkalian}\\
        &=(-12y-12xy')y'-12xyy''
    \end{align*}
    \item \[
        \drv{x}{6y^{2}y'} = \drv{x}{6y^{2}}y' + 6y^{2}\drv{x}{y'} 
        =\bracket*{12y\drv{x}{y}}y'+6y^{2}y''=12y(y')^{2}+6y^{2}y''
    \]
    \end{enumerate}
    Diperoleh
    \begin{align*}
        &\drv{x}{3x^{2}}+\drv{x}{6xy}+\drv{x}{3x^{2}y'}
        +\drv{x}{-6y^{2}}+\drv{x}{-12xyy'}+\drv{x}{6y^{2}y'}=0\\
        \iff &(6x)+(6y+6xy')+(6xy'+3x^{2}y'')+(-12yy')
        +\bracket*{(-12y-12xy')y'-12xyy''}\\
        &+(12y(y')^{2}+6y^{2}y'')=0
    \end{align*}
    Subtitusi $x=1$, $y=1$, dan $y'=1$ diperoleh
    \begin{align*}
        &(6x)+(6y+6xy')+(6xy'+3x^{2}y'')+(-12yy')
        +\bracket*{(-12y-12xy')y'-12xyy''}\\
        &+(12y(y')^{2}+6y^{2}y'')=0\\
        \iff &(6(1))+(6(1)+6(1)(1))+(6(1)(1)+3(1)^{2}y'')+(-12(1)(1))\\
        &+\bracket*{(-12(1)-12(1)(1))(1)-12(1)(1)y''}+(12(1)(1)^{2}+6(1)^{2}y'')=0\\
        \iff &6+6+6+6+3y''-12-12-12-12y''+12+6y''=0\\
        \iff &-3y''=0 \iff y''=0
    \end{align*}
    Diperoleh $\ds\frac{d^{2}y}{dx^{2}}$ dari 
    $x^{3}+3x^{2}y-6xy^{2}+2y^{3}=0$ pada titik $(1,1)$ adalah $0$.
    
    $\therefore$ $\ds\frac{dy}{dx}$ dan $\ds\frac{d^{2}y}{dx^{2}}$ dari 
    $x^{3}+3x^{2}y-6xy^{2}+2y^{3}=0$ pada titik $(1,1)$ adalah $1$ dan $0$.
\begin{center}
    \line(1,0){150}
\end{center}
    \item Pertama akan dicari $\ds\frac{dy}{dx}$ dari 
    $y^{4}+\cos\bracket*{x^{2}y^{3}}+3y^{2}+x^{2}=5$ pada titik $(0,1)$.

    Turunkan secara implisit:
    \begin{align*}
        &\drv{x}{y^{4}+\cos\bracket*{x^{2}y^{3}}+3y^{2}+x^{2}}=\drv{x}{5}\\
        \iff &\drv{x}{y^{4}}+\drv{x}{\cos\bracket*{x^{2}y^{3}}}
        +\drv{x}{3y^{2}}+\drv{x}{x^{2}} = 0
    \end{align*}
    Selesaikan masing-masing turunan:
    \begin{enumerate}[label={\arabic*})]
    \item \[
    \drv{x}{y^{4}} = 4y^{3}\drv{x}{y} = 4y^{3}y'
    \]
    \item \begin{align*}
        \drv{x}{\cos\bracket*{x^{2}y^{3}}} 
        &= -\sin\bracket*{x^{2}y^{3}}\drv{x}{x^{2}y^{3}}
        &\text{aturan rantai}\\
        &= -\sin\bracket*{x^{2}y^{3}}\bracket*{\drv{x}{x^{2}}y^{3}+x^{2}\drv{x}{y^{3}}}
        &\text{aturan perkalian}\\
        &= -\sin\bracket*{x^{2}y^{3}}\bracket*{2xy^{3}+x^{2}\bracket*{3y^{2}\drv{x}{y}}}
        &\text{aturan rantai}\\
        &= -\sin\bracket*{x^{2}y^{3}}(2xy^{3}+3x^{2}y^{2}y')
    \end{align*}
    \item \[
    \drv{x}{3y^{2}} = 6y\drv{x}{y} = 6yy'
    \]
    \item \[
    \drv{x}{x^{2}} = 2x
    \]
    \end{enumerate}
    Diperoleh
    \begin{align*}
        &\drv{x}{y^{4}}+\drv{x}{\cos\bracket*{x^{2}y^{3}}}
        +\drv{x}{3y^{2}}+\drv{x}{x^{2}} = 0\\
        \iff &4y^{3}y'
        +\bracket*{-\sin\bracket*{x^{2}y^{3}}(2xy^{3}+3x^{2}y^{2}y')}+6yy'+2x=0
    \end{align*}
    Subtitusi $x=0$ dan $y=1$ diperoleh
    \begin{align*}
        &4y^{3}y'
        +\bracket*{-\sin\bracket*{x^{2}y^{3}}(2xy^{3}+3x^{2}y^{2}y')}+6yy'+2x=0\\
        \iff &4(1)^{3}y'
        +\bracket*{-\sin\bracket*{(0)^{2}(1)^{3}}(2(0)(1)^{3}+3(0)^{2}(1)^{2}y')}+6(1)y'
        +2(0)=0\\
        \iff &4y'+(-\sin(0)(0+0))+6y'+0=0 \\
        \iff &10y'=0 \iff y'=0
    \end{align*}
    Diperoleh $\ds\frac{dy}{dx}$ dari 
    $y^{4}+\cos\bracket*{x^{2}y^{3}}+3y^{2}+x^{2}=5$ pada titik $(0,1)$ adalah $0$.

    \item Kedua akan dicari $\ds\frac{d^{2}y}{dx^{2}}$ dari 
    $y^{4}+\cos\bracket*{x^{2}y^{3}}+3y^{2}+x^{2}=5$ pada titik $(0,1)$.

    Sebelumnya telah diperoleh
    \[
    4y^{3}y'+\bracket*{-\sin\bracket*{x^{2}y^{3}}(2xy^{3}+3x^{2}y^{2}y')}+6yy'+2x=0
    \]
    Turunkan secara implisit:
    \begin{align*}
        &\drv{x}{4y^{3}y'
        +\bracket*{-\sin\bracket*{x^{2}y^{3}}(2xy^{3}+3x^{2}y^{2}y')}+6yy'+2x}=
        \drv{x}{0}\\
        \iff &\drv{x}{4y^{3}y'}+\drv{x}{-\sin\bracket*{x^{2}y^{3}}(2xy^{3}+3x^{2}y^{2}y')}
        +\drv{x}{6yy'}\\
        &+\drv{x}{2x}=0
    \end{align*}
    Selesaikan masing-masing turunan:
    \begin{enumerate}[label={\arabic*})]
    \item \begin{align*}
        \drv{x}{4y^{3}y'} &= \drv{x}{4y^{3}}y' + 4y^{3}\drv{x}{y'}
        &\text{aturan perkalian}\\
        &=\bracket*{12y^{2}\drv{x}{y}}y'+4y^{3}y''
        &\text{aturan rantai}\\
        &=12y^{2}(y')^{2}+4y^{3}y''
    \end{align*}
    \item \begin{align*}
        \drv{x}{6yy'} &= \drv{x}{6y}y'+6y\drv{x}{y'}
        &\text{aturan perkalian}\\
        &=6\drv{x}{y}y'+6yy'' \\
        &= 6(y')^{2}+6yy''
    \end{align*}
    \item \[
    \drv{x}{2x} = 2
    \]
    \item \begin{align*}
        &\drv{x}{-\sin\bracket*{x^{2}y^{3}}(2xy^{3}+3x^{2}y^{2}y')}\\
        =\,&\drv{x}{-\sin\bracket*{x^{2}y^{3}}}(2xy^{3}+3x^{2}y^{2}y')
        +\bracket*{-\sin\bracket*{x^{2}y^{3}}}\drv{x}{2xy^{3}+3x^{2}y^{2}y'}
        \quad *\\
        =\,&\bracket*{-\cos\bracket*{x^{2}y^{3}}\bracket*{2xy^{3}+x^{2}3y^{2}y'}}
        (2xy^{3}+3x^{2}y^{2}y')\\
        &+\bracket*{-\sin\bracket*{x^{2}y^{3}}}
        \bracket*{2y^{3}+12xy^{2}y'+6x^{2}y(y')^{2}+3x^{2}y^{2}y''}\\
        =\,&\bracket*{-\cos\bracket*{x^{2}y^{3}}\bracket*{2xy^{3}+x^{2}3y^{2}y'}^{2}}
        \\&+\bracket*{-\sin\bracket*{x^{2}y^{3}}}
        \bracket*{2y^{3}+12xy^{2}y'+6x^{2}y(y')^{2}+3x^{2}y^{2}y''}
    \end{align*}
    $*$ Selesaikan masing-masing turunan ini
    \begin{align*}
        \drv{x}{-\sin\bracket*{x^{2}y^{3}}}
        &=-\cos\bracket*{x^{2}y^{3}}\drv{x}{x^{2}y^{3}}
        &\text{aturan rantai}\\
        &=-\cos\bracket*{x^{2}y^{3}}
        \bracket*{\drv{x}{x^{2}}y^{3}+x^{2}\drv{x}{y^{3}}}
        &\text{aturan perkalian}\\
        &=-\cos\bracket*{x^{2}y^{3}}
        \bracket*{2xy^{3}+x^{2}3y^{2}\drv{x}{y}}
        &\text{aturan rantai}\\
        &=-\cos\bracket*{x^{2}y^{3}}
        \bracket*{2xy^{3}+x^{2}3y^{2}y'}
    \end{align*}
    dan
    \begin{align*}
        &\drv{x}{2xy^{3}+3x^{2}y^{2}y'} \\
        =\,&\drv{x}{2xy^{3}}+\drv{x}{3x^{2}y^{2}y'}\\
        =\,&\bracket*{\drv{x}{2x}y^{3}+2x\drv{x}{y^{3}}}
        +\bracket*{\drv{x}{3x^{2}y^{2}}y'+3x^{2}y^{2}\drv{x}{y'}}\\
        =\,&\bracket*{2y^{3}+2x(3y^2)\drv{x}{y}}
        +\bracket*{\bracket*{\drv{x}{3x^{2}}y^2+3x^{2}\drv{x}{y^2}}y'
        +3x^{2}y^{2}y''}\\
        &=(2y^{3}+6xy^{2}y')+\bracket*{\bracket*{6xy^{2}+3x^{2}2y\drv{x}{y}}y'
        +3x^{2}y^{2}y''}\\
        &=2y^{3}+6xy^{2}y'+6xy^{2}y'+6x^{2}y(y')^{2}+3x^{2}y^{2}y''\\
        &=2y^{3}+12xy^{2}y'+6x^{2}y(y')^{2}+3x^{2}y^{2}y''\\
    \end{align*}
    \end{enumerate}
    Diperoleh
    \begin{align*}
        &\drv{x}{4y^{3}y'}+\drv{x}{-\sin\bracket*{x^{2}y^{3}}(2xy^{3}+3x^{2}y^{2}y')}
        +\drv{x}{6yy'}\\
        &+\drv{x}{2x}=0\\
        \iff &\bracket*{12y^{2}(y')^{2}+4y^{3}y''}
        +\bracket*{-\cos\bracket*{x^{2}y^{3}}\bracket*{2xy^{3}+x^{2}3y^{2}y'}^{2}}
        \\&+\bracket*{-\sin\bracket*{x^{2}y^{3}}}
        \bracket*{2y^{3}+12xy^{2}y'+6x^{2}y(y')^{2}+3x^{2}y^{2}y''}\\
        &+\bracket*{6(y')^{2}+6yy''} + 2 = 0
    \end{align*}
    Subtitusi $x=0$, $y=1$, dan $y'=0$ diperoleh
    \begin{align*}
        &\bracket*{12y^{2}(y')^{2}+4y^{3}y''}
        +\bracket*{-\cos\bracket*{x^{2}y^{3}}\bracket*{2xy^{3}+x^{2}3y^{2}y'}^{2}}
        \\&+\bracket*{-\sin\bracket*{x^{2}y^{3}}}
        \bracket*{2y^{3}+12xy^{2}y'+6x^{2}y(y')^{2}+3x^{2}y^{2}y''}\\
        &+\bracket*{6(y')^{2}+6yy''} + 2 = 0\\
        \iff &\bracket*{12(1)^{2}(0)^{2}+4(1)^{3}y''}
        +\bracket*{-\cos\bracket*{(0)^{2}(1)^{3}}\bracket*{2(0)(1)^{3}+(0)^{2}3(1)^{2}(1)}^{2}}
        \\&+\bracket*{-\sin\bracket*{(0)^{2}(1)^{3}}}
        \bracket*{2(1)^{3}+12(0)(1)^{2}(0)'+6(0)^{2}(1)(0)^{2}+3(0)^{2}(1)^{2}y''}\\
        &+\bracket*{6(0)^{2}+6(1)y''} + 2 = 0\\
        \iff &(0+4y'')+(-\cos(0)(0))+(-\sin(0)(2))+(0+6y'')+2=0\\
        \iff &10y''+2=0 \iff y'' = -\frac{1}{5}
    \end{align*}
    Diperoleh $\ds\frac{d^{2}y}{dx^{2}}$ dari 
    $y^{4}+\cos\bracket*{x^{2}y^{3}}+3y^{2}+x^{2}=5$ pada titik $(0,1)$ adalah $\ds-\frac{1}{5}$.
    
    $\therefore$ $\ds\frac{dy}{dx}$ dan $\ds\frac{d^{2}y}{dx^{2}}$ dari 
    $y^{4}+\cos\bracket*{x^{2}y^{3}}+3y^{2}+x^{2}=5$ pada titik $(0,1)$ adalah $0$ dan $\ds-\frac{1}{5}$.
\end{enumerate}
\vspace{0.1cm}
\textbf{Catatan:}\\
Cara lebih cepat untuk menyelesaikan soal ini, khususnya saat mencari 
turunan kedua soal 1b adalah saat ketika mencari turunan masing-masing, 
lakukan subtitusi nilai $x$, $y$ dan $y'$ pada baris dengan simbol $*$. 
Hal ini dikarenakan beberapa nilainya akan berujung $0$ sehingga penurunan 
hanya akan membuang waktu.
\begin{center}
    \line(1,0){300}
\end{center}

\end{enumerate}

\begin{center}
    \line(1,0){300}
\end{center}