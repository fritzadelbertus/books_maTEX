\newpage
\begin{flushright}
    \textbf{\Large{Ujian Akhir Semester}}
    \subsection*{Tahun 2023}
    \addcontentsline{toc}{subsection}{UAS - 2023}
\end{flushright}
\vspace{0.5cm}
\hrule height 2pt
\vspace{0.5cm}
\begin{center}
    \textbf{\large{MATERI}}
    \begin{enumerate}[leftmargin=*, label={\arabic*}.]
        \item Aplikasi integral dalam mencari volume benda putar
        \item Fungsi Logaritma dan Ekponensial
        \item Fungsi Invers dan Turunan Invers
        \item Teknik-teknik Integrasi
    \end{enumerate}
\end{center}
\vspace{0.2cm}
\hrule height 1pt
\vspace{0.5cm}
\begin{center}
    \textbf{\large{SOAL}}
\end{center}
\begin{enumerate}[leftmargin=*, label={\arabic*}.]
\item Diberikan suatu daerah tertutup di kuadran $I$ yang dibatasi oleh $y=x^{2}$, 
$y=2-x$, dan sumbu-$y$.
\begin{enumerate}[label={\alph*}.]
    \item Sketsalah daerah tertutup tersebut!
    \item Hitunglah volume benda putar jika daerah tertutup tersebut 
    diputar mengelilingi
    \begin{enumerate}[label={\roman*}.]
        \item Garis $y=-1$.
        \item Garis $x=3$.
    \end{enumerate}
\end{enumerate}
\item Selesaikanlah
\begin{enumerate}[label={\alph*}.]
    \item Jika $\ds f(x) = \ln\brk*{\frac{ax-b}{ax+b}}^{c}$ dengan 
    $\ds c=\frac{a^{2}-b^{2}}{2ab}$, maka tentukanlah $f'(1)$.
    \item Misalkan $f(x)=\int_0^{x}\sqrt{1+\cos^{2}t}\,dt$, tentukanlah 
    nilai $(f^{-1})'(0)$.
\end{enumerate}
\item Selesaikanlah
\begin{enumerate}[label={\alph*}.]
    \item $\int \ln (x+x^{2})\,dx$
    \item $\int \tan^{3}(3y)\sec^{3}(3y)\,dy$
\end{enumerate}
\item Tentukanlah
\begin{enumerate}[label={\alph*}.]
    \item $\ds \int \frac{x}{x^{2}-4x+8}\,dx$
    \item $\ds \int \frac{x^{2}+x-2}{3x^{3}-x^{2}+3x-1}\,dx$
\end{enumerate}
\end{enumerate}
\vspace{0.2cm}
\hrule height 1pt
\vspace{0.5cm}
\begin{center}
    \textbf{\large{PEMBAHASAN}}
\end{center}
\begin{enumerate}[leftmargin=*, label={\arabic*}.]
\item Diberikan suatu daerah tertutup di kuadran $I$ yang dibatasi oleh $y=x^{2}$, 
$y=2-x$, dan sumbu-$y$.
\begin{enumerate}[label={\alph*}.]
    \item Akan disketsa daerah tertutup tersebut.\\
    Uji dua titik untuk sketsa $y=2-x$ dan tiga titik untuk $y=x^2$.
    \begin{center}
        \begin{tabular}{|c|c|c|}\hline
            $x$  & $0$ & $1$ \\ \hline
            $y=2-x$ & $2$ & $1$ \\ \hline
        \end{tabular}\quad
        \begin{tabular}{|c|c|c|c|}\hline
            $x$ & $-1$ & $0$ & $1$ \\ \hline
            $y=x^{2}$ & $1$ & $0$ & $1$ \\ \hline
        \end{tabular}
    \end{center}
    Diperoleh daerah yang dibatasi seperti berikut.
\begin{center}
    \begin{tikzpicture}[>=stealth]
    \begin{axis}[
        xmin=-2.5,xmax=3.5,
        ymin=-1.5,ymax=3,
        axis x line=middle,
        axis y line=middle,
        axis line style=<->,
        xlabel={$x$},
        ylabel={$y$},
        ]
        \addplot [name path=f, no marks,blue, domain=-2:4,samples=50]({x},{2-x});
        \node [blue] at (2,0.8){\scalebox{0.7}{$y=2-x$}};
        \addplot [name path=g, no marks,red, domain=-2:4,samples=50]({x},{x^2});
        \node [red] at (-1.5,1){\scalebox{0.7}{$y=x^{2}$}};
        \addplot [no marks,black, domain=-3:4,samples=50]({x},{-1});
        \node [black] at (1,-0.8){\scalebox{0.7}{$y=-1$}};
        \addplot [no marks,black, domain=-2:3,samples=50]({3},{x});
        \node [black] at (2.5,2.5){\scalebox{0.7}{$x=3$}};
        \node [draw, shape = circle, fill = blue, minimum size = 0.1cm, inner sep=0pt] at (0,2){};
        \node [draw, shape = circle, fill = purple, minimum size = 0.1cm, inner sep=0pt] at (1,1){};
        \node [draw, shape = circle, fill = red, minimum size = 0.1cm, inner sep=0pt] at (-1,1){};
        \node [draw, shape = circle, fill = red, minimum size = 0.1cm, inner sep=0pt] at (0,0){};
        \addplot [thick, color=cyan, fill=cyan, fill opacity=0.5]
        fill between[of=f and g, soft clip={domain=0:1},]; 
    \end{axis}
    \end{tikzpicture}
\end{center}

$\therefore$ Telah disketsa daerah tertutup tersebut.
\begin{center}
    \line(1,0){150}
\end{center}
    \item Sebelum mencari volume, akan dicari titik potong $y=2-x$ dan $y=x^{2}$.
    \begin{align*}
        y=2-x,\,y=x^{2} &\Longrightarrow x^{2}=2-x\\
        &\iff x^{2}+x-2 = 0\\
        &\iff (x+2)(x-1) = 0\\
    \end{align*}
    Diperoleh titik potong saat $x=-2$ dan $x=1$ ($x=1$ menjadi batas partisi)
    \begin{enumerate}[label={\roman*}.]
        \item Akan dicari volume benda putar ketika daerah tertutup tersebut 
        diputar mengelilingi \\garis $y=-1$.

        Gunakan metode cincin, partisi tegak lurus garis $y=1$ dari $x=0$ 
        sampai $x=1$.\\
        Jari-jari luar cincin adalah jarak $y=2-x$ ke $y=-1$ yaitu $2-x+1=3-x$.\\
        Jari-jari dalam cincin adalah jarak $y=x^2$ ke $y=-1$ yaitu $x^{2}+1$.

        Gunakan integral untuk memperoleh volume
        \begin{align*}
            V &= \int \text{Luas bagian luar} - \text{Luas bagian dalam}\\
            &=\int_{0}^{1} \pi(3-x)^2 - \pi(x^{2}+1)^{2}\,dx \\
            &=\pi\int_{0}^{1} \brk*{x^{2}-6x+9} - \brk*{x^{4}+2x^{2}+1}\,dx \\
            &=\pi\int_{0}^{1} -x^{4}-x^{2}-6x+8\,dx \\
        \end{align*}
        \begin{align*}
            &=\pi\eval{-\frac{1}{5}x^{5}-\frac{1}{3}x^{3}-3x^{2}+8x}{0}{1}\\
            &=\pi\brk*{\brk*{-\frac{1}{5}(1)^{5}
            -\frac{1}{3}(1)^{3}-3(1)^{2}+8(1)}-(0)}\\
            &=\pi\brk*{5-\frac{8}{15}}\\
            &=\frac{67}{15}\pi
        \end{align*}
        Berikut ilustrasi hasil benda padat
\begin{center}
\begin{tikzpicture}
    \begin{axis}[view={-30}{30},colormap/Spectral-9]
    \begin{scope}
        \def\ry{(-1)}
        \def\fx{(2-x)}
        \addplot3[surf,shader=flat,name path=f,
            samples=25,
            domain=0:1,y domain=0:2*pi,
            z buffer=sort]
            (x,{((\fx-\ry) * cos(deg(y))+\ry)}, {((\fx-\ry) * sin(deg(y))+\ry)});
    \end{scope}
    \begin{scope}
        \def\ry{(-1)}
        \def\fx{(x^2)}
        \addplot3[surf,shader=flat,name path=g,
            samples=25,
            domain=0:1,y domain=0:2*pi,
            opacity=0.75,
            z buffer=sort]
            (x,{((\fx-\ry) * cos(deg(y))+\ry)}, {((\fx-\ry) * sin(deg(y))+\ry)});
    \end{scope}
    \addplot3[surf,shader=flat,name path=g,
        samples=25,
        domain=0:2,y domain=0:2*pi,
        z buffer=sort]
        (0,{(x+1)*cos(deg(y))-1},{(x+1)*sin(deg(y))-1});
    \addplot3[black,shader=flat,
            samples=25,
            domain=0:1,
            z buffer=sort]
            (x,-1,-1);
    \node at (0,-1,-1) {$y=-1$};
    \end{axis}
\end{tikzpicture}
\end{center}
$\therefore$ Diperoleh volume benda putar terhadap garis 
$y=-1$ adalah $\ds\frac{67}{15}\pi$.
\begin{center}
    \line(1,0){150}
\end{center}
        \item Akan dicari volume benda putar ketika daerah tertutup tersebut 
        diputar mengelilingi \\garis $x=3$.

        Gunakan metode kulit tabung, partisi sejajar garis $x=3$.\\
        Jari-jari kulit tabung adalah $3-x$ untuk $0 \leq x \leq 1$.\\
        Tinggi kulit tabung adalah jarak $y=x^{2}$ ke $y=2-x$ yaitu $2-x-x^{2}$.

        Gunakan integral untuk memperoleh volume
        \begin{align*}
            V &= \int \text{Luas Kulit Tabung}\\
            &=\int_{0}^{1} 2\pi(\text{jari-jari})(\text{tinggi})\,dx\\
            &=2\pi\int_{0}^{1} (3-x)\brk*{2-x-x^{2}}\,dx\\
            &=2\pi\int_{0}^{1} \brk*{6-3x-3x^{2}-2x+x^{2}+x^{3}}\,dx\\
            &=2\pi\int_{0}^{1} \brk*{x^{3}-2x^{2}-5x+6}\,dx
        \end{align*}
        \begin{align*}
            &=2\pi\eval{\frac{1}{4}x^{4}-\frac{2}{3}x^{3}-\frac{5}{2}x^{2}+6x}{0}{1}\\
            &=2\pi\brk*{\brk*{\frac{1}{4}(1)^{4}-\frac{2}{3}(1)^{3}
            -\frac{5}{2}(1)^{2}+6(1)}-(0)}\\
            &=2\pi\frac{37}{12}\\
            &=\frac{37}{6}\pi
        \end{align*}
    Berikut ilustrasi hasil benda padat
\begin{center}
\begin{tikzpicture}
    \begin{axis}[view={-30}{30},colormap/Spectral-9]
    \begin{scope}
        \def\rx{(3)}
        \def\fx{(x^2)}
        \addplot3[surf,shader=flat,name path=g,
            samples=25,
            domain=0:1,y domain=0:2*pi,
            z buffer=sort]
            ({(((x-\rx) * cos(deg(y)))+\rx)}, {(((x-\rx) * sin(deg(y)))+\rx)}, {\fx});
    \end{scope}
    \begin{scope}
        \def\rx{(3)}
        \def\fx{(2-x)}
        \addplot3[surf,shader=flat,name path=f,
            samples=25,
            domain=0:1,y domain=0:2*pi,
            z buffer=sort]
            ({(((x-\rx) * cos(deg(y)))+\rx)}, {(((x-\rx) * sin(deg(y)))+\rx)}, {\fx});
    \end{scope}
    \addplot3[surf,shader=flat,name path=g,
        samples=25,
        domain=0:2,y domain=0:2*pi,
        opacity=0.25,
        z buffer=sort]
        ({3*cos(deg(y))+3},{3*sin(deg(y))+3},x);
    \addplot3[black,shader=flat,
            samples=25,
            domain=0:2,
            z buffer=sort]
            (3,3,x);
    \node at (3,1,2) {$x=3$};
    \end{axis}
\end{tikzpicture}
\end{center}
$\therefore$ Diperoleh volume benda putar terhadap garis 
$x=3$ adalah $\ds\frac{37}{6}\pi$.
    \end{enumerate}
\end{enumerate}
\begin{center}
    \line(1,0){300}
\end{center}
\item 
\begin{enumerate}[label={\alph*}.]
    \item Diberikan $\ds f(x) = \ln\brk*{\frac{ax-b}{ax+b}}^{c}$ dengan 
    $\ds c=\frac{a^{2}-b^{2}}{2ab}$.\\
    Akan dicari $f'(1)$.

    Turunakan $f$.
    \begin{align*}
        &f(x) = \ln\brk*{\frac{ax-b}{ax+b}}^{c}\\
        \iff &f(x) = c\brk*{\ln(ax-b)-\ln(ax+b)}\\
        &f'(x) = \drv{x}{c\brk*{\ln(ax-b)-\ln(ax+b)}}\\
        \iff &f'(x) = c\drv{x}{\ln(ax-b)-\ln(ax+b)}\\
        \iff &f'(x) = c\brk*{\frac{1}{ax-b}\drv{x}{ax-b}
        -\frac{1}{ax+b}\drv{x}{ax+b}}\\
        \iff &f'(x) = c\brk*{\frac{a}{ax-b}
        -\frac{a}{ax+b}}\\
        &\text{subtitusi $x=1$}\\
        &f'(1) = c\brk*{\frac{a}{a(1)-b}-\frac{a}{a(1)+b}}\\
        \iff &f'(1) = c\brk*{\frac{a}{a-b}-\frac{a}{a+b}}\\
        \iff &f'(1) = c\brk*{\frac{a(a+b)-a(a-b)}{(a-b)(a+b)}}\\
        \iff &f'(1) = c\brk*{\frac{a^{2}+ab-a^{2}+ab}{a^{2}-b^{2}}}\\
        \iff &f'(1) = c\brk*{\frac{2ab}{a^{2}-b^{2}}}\\
        &\text{subtitusi nilai $c$}\\
        \iff &f'(1) = \frac{a^{2}-b^{2}}{2ab}\brk*{\frac{2ab}{a^{2}-b^{2}}}\\
        \iff &f'(1) = 1
    \end{align*}
    $\therefore$ Diperoleh $f'(1)=1$.
\begin{center}
    \line(1,0){150}
\end{center}
    \item Diberikan $f(x)=\int_0^{x}\sqrt{1+\cos^{2}t}\,dt$.\\
    Akan dicari nilai $(f^{-1})'(0)$.\\
    Gunakan teorema turunan invers yaitu
    \[
    (f^{-1})'(y) = \frac{1}{f'(x)}
    \]
    Sehingga perlu dicari nilai $x$ sedemikian sehingga $f(x)=y=0$, dalam kasus 
    ini berarti mencari nilai $x$ sehingga $\int_0^{x}\sqrt{1+\cos^{2}t}\,dt = 0$. 
    Karena $x$ batas atas maka cukup menyamakannya dengan batas bawah sehingga 
    integralnya pasti bernilai $0$ yaitu $\int_0^{0}\sqrt{1+\cos^{2}t}\,dt = 0$.\\
    Dengan demikian
    \[
    (f^{-1})'(0) = \frac{1}{f'(0)}
    \]
    Akan dicari $f'(0)$.
    \begin{align*}
        f(x)&=\int_0^{x}\sqrt{1+\cos^{2}t}\,dt\\
        f'(x)&=\drv{x}{\int_0^{x}\sqrt{1+\cos^{2}t}\,dt}\\
        &=\sqrt{1+\cos^{2}x}\\
        &\text{subtitusi $x=0$}\\
        f'(0) &= \sqrt{1+\cos^{2}(0)} = \sqrt{1+1^{2}} = \sqrt{2}
    \end{align*}
    Sehingga 
    \[
    (f^{-1})'(0) = \frac{1}{f'(0)} = \frac{1}{\sqrt{2}} = \frac{\sqrt{2}}{2}
    \]
    $\therefore$ Diperoleh $\ds (f^{-1})'(0)=\frac{\sqrt{2}}{2}$
\end{enumerate}
\begin{center}
    \line(1,0){300}
\end{center}
\item 
\begin{enumerate}[label={\alph*}.]
    \item Akan dicari $\ds\int \ln (x+x^{2})\,dx$
    
    Gunakan metode subtitusi dan integral parsial
    \begin{align*}
        &\int \ln (x+x^{2})\,dx\\
        &\text{integral parsial dengan $u=\ln(x+x^{2})$ dan $dv=dx$}\\
        =\,&\ln(x+x^{2})(x)-\int(x)\frac{1}{x+x^{2}}\brk*{\drv{x}{x+x^{2}}}\,dx\\
        =\,&x\ln(x+x^{2})-\int \frac{1+2x}{1+x}\,dx\\
        =\,&x\ln(x+x^{2})-\int \frac{2+2x-1}{1+x}\,dx\\
        =\,&x\ln(x+x^{2})-\brk*{\int \frac{2+2x}{1+x}\,dx-\int\frac{1}{1+x}\,dx}\\
        &\text{subtitusi $u=1+x$ dan $du=dx$}\\
        =\,&x\ln(x+x^{2})-\brk*{\int 2\,dx-\int\frac{1}{u}\,du}\\
        =\,&x\ln(x+x^{2})-\brk*{2x-\ln\abs{u}}+C\\
        &\text{subtitusi balik $u=1+x$}\\
        =\,&x\ln(x+x^{2})-2x+\ln\abs{1+x}+C\\
    \end{align*}
    $\therefore$ Diperoleh $\ds \int \ln (x+x^{2})\,dx
    =x\ln(x+x^{2})-2x+\ln\abs{1+x}+C$
\begin{center}
    \line(1,0){150}
\end{center}
    \item Akan dicari $\ds\int \tan^{3}(3y)\sec^{3}(3y)\,dy$
    
    Gunakan metode subtitusi dan identitas trigonometri.
    \begin{align*}
        &\int \tan^{3}(3y)\sec^{3}(3y)\,dy\\
        =\,&\frac{3}{3}\int \tan^{2}(3y)\sec^{2}(3y)\tan(3y)\sec(3y)\,dy\\
        =\,&\frac{1}{3}\int \brk*{\sec^{2}(3y)-1}\sec^{2}(3y)3\tan(3y)\sec(3y)\,dy\\
        &\text{subtitusi $u=\sec(3y)$ dan $du=3\sec(3y)\tan(3y)\,dy$}\\
        =\,&\frac{1}{3}\int \brk*{u^{2}-1}u^{2}\,du\\
        =\,&\frac{1}{3}\int u^{4}-u^{2}\,du\\
        =\,&\frac{1}{3}\brk*{\frac{1}{5}u^{5}-\frac{1}{3}u^{3}}+C\\
        &\text{subtitusi balik $u=\sec(3y)$}\\
        =\,&\frac{\sec^{5}(3y)}{15}-\frac{\sec^{3}(3y)}{9}+C
    \end{align*}
    $\therefore$ Diperoleh $\ds\int \tan^{3}(3y)\sec^{3}(3y)\,dy
    =\frac{\sec^{5}(3y)}{15}-\frac{\sec^{3}(3y)}{9}+C$
\end{enumerate}
\begin{center}
    \line(1,0){300}
\end{center}
\item 
\begin{enumerate}[label={\alph*}.]
    \item Akan dicari $\ds \int \frac{x}{x^{2}-4x+8}\,dx$
    
    Ubah bentuk integral dan lakukan subtitusi trigonometri
    \begin{align*}
        &\int \frac{x}{x^{2}-4x+8}\,dx\\
        =\,&\int \frac{x}{\brk*{x^{2}-4x+4}+4}\,dx\\
        =\,&\int \frac{(x-2)+2}{(x-2)^{2}+2^2}\,dx\\
        &\text{subtitusi $x-2 = 2\tan u$ dan $dx = 2\sec^{2}u\,du$}\\
        =\,&\int \frac{2\tan u+2}{(2\tan u)^{2}+2^2}2\sec^{2}u\,du\\
        =\,&\int \frac{2\tan u+2}{(2\tan u)^{2}+2^2}2\sec^{2}u\,du\\
        =\,&\int 4\frac{\tan u+1}{2^{2}\brk*{\tan^{2}u+1}}\sec^{2}u\,du\\
    \end{align*}
    \begin{align*}
        =\,&\int \frac{\tan u+1}{\sec^{2}u}\sec^{2}u\,du\\
        =\,&\int \tan u+1\,du\\
        =\,&-\ln\abs{\cos u} + u+C\\
    \end{align*}
Karena $x-2 = 2\tan u$ maka dengan aturan segitiga
\begin{align*}
    \tan u &= \frac{x-2}{2}\\
    u &= \arctan\brk*{\frac{x-2}{2}}\\
    \cos u &= \frac{2}{\sqrt{(x-2)^{2}+2^{2}}} = 2\brk*{x^{2}-4x+8}^{-1/2}
\end{align*}
Sehingga 
\begin{align*}
    &\int \frac{x}{x^{2}-4x+8}\,dx\\
    =\,&=-\ln\abs{\cos u} + u+C\\
    =\,&-\ln\abs*{2\brk*{x^{2}-4x+8}^{-1/2}} + \arctan\brk*{\frac{x-2}{2}}+C\\
    =\,&-\brk*{\ln 2 + \ln\abs*{\brk*{x^{2}-4x+8}^{-1/2}}} 
    + \arctan\brk*{\frac{x-2}{2}}+C\\
    =\,&-\ln\abs*{\brk*{x^{2}-4x+8}^{-1/2}} 
    + \arctan\brk*{\frac{x-2}{2}}+C-\ln 2\\
    =\,&\frac{1}{2}\ln\abs*{x^{2}-4x+8}+\arctan\brk*{\frac{x-2}{2}}+C\\
\end{align*}
$\therefore$ Diperoleh $\ds\int \tan^{3}(3y)\sec^{3}(3y)\,dy
    =\frac{\sec^{5}(3y)}{15}-\frac{\sec^{3}(3y)}{9}+C$
\begin{center}
    \line(1,0){150}
\end{center}
    \item Akan dicari $\ds \int \frac{x^{2}+x-2}{3x^{3}-x^{2}+3x-1}\,dx$
    
    Perhatikan bahwa 
    \begin{align*}
        \frac{x^{2}+x-2}{3x^{3}-x^{2}+3x-1} 
        &= \frac{x^{2}+x-2}{3x^{3}+3x-x^{2}-1}\\
        &= \frac{x^{2}+x-2}{3x(x^{2}+1)-(x^{2}+1)}\\
        &= \frac{x^{2}+x-2}{(3x-1)(x^{2}+1)}\\
    \end{align*}
    Lakukan dekomposisi pecahan
    \begin{align*}
        &\frac{x^{2}+x-2}{(3x-1)(x^{2}+1)} 
        = \frac{A}{3x-1}+\frac{Bx+C}{x^{2}+1}\\
        \iff & x^{2}+x-2 = A(x^{2}+1)+(Bx+C)(3x-1)\\
        \iff & x^{2}+x-2 = Ax^{2}+A+(3Bx^{2}-Bx+3Cx-C)\\
        \iff & x^{2}+x-2 = (A+3B)x^{2}+(3C-B)x+(A-C)\\
    \end{align*}
    Diperoleh SPL
    \begin{center}
    \begin{tabular}{llll}
        $A$ &$+3B$ & &$=1$\\
        $A$ &$-B$ &$+3C$ &$=1$\\
        $A$ & &$-C$ &$=-2$
    \end{tabular}
    \end{center}
    Dengan solusi 
    \[
    \ds A=\frac{-7}{5},\quad B=\frac{4}{5},\quad C=\frac{3}{5}
    \]
    Sehingga
    \[
        \frac{x^{2}+x-2}{(3x-1)(x^{2}+1)} 
        = \frac{-7/5}{3x-1}+\frac{(4/5)x+(3/5)}{x^{2}+1}
        = \frac{1}{5}\brk*{\frac{-7}{3x-1}+\frac{4x+3}{x^{2}+1}}
    \]
    Selesaikan integral
    \begin{align*}
        &\int \frac{x^{2}+x-2}{3x^{3}-x^{2}+3x-1}\,dx\\
        =\,&\frac{1}{5}\int\frac{-7}{3x-1}+\frac{4x+3}{x^{2}+1}\,dx\\
        =\,&\frac{1}{5}\brk*{-7\int\frac{1}{3x-1}\,dx
        +\int\frac{4x+3}{x^{2}+1}\,dx}\\
        =\,&\frac{1}{5}\brk*{-7\int\frac{1}{3x-1}\,dx
        +\int\frac{4x}{x^{2}+1}\,dx+\int\frac{3}{x^{2}+1}\,dx}\\
        =\,&\frac{1}{5}\brk*{-\frac{7}{3}\int\frac{1}{3x-1}3\,dx
        +2\int\frac{1}{x^{2}+1}2x\,dx+3\arctan x}\\
        &\text{subtitusi $u=3x-1$, $du=3\,dx$, $w=x^{2}+1$, dan $dw=2x\,dx$}\\
        =\,&\frac{1}{5}\brk*{-\frac{7}{3}\int\frac{1}{u}\,du
        +2\int\frac{1}{w}\,dw+3\arctan x}\\
        =\,&\frac{1}{5}\brk*{-\frac{7}{3}\ln\abs{u}
        +2\ln\abs{w}+3\arctan x}+C\\
        &\text{subtitusi balik $u=3x-1$ $w=x^{2}+1$}\\
        =\,&-\frac{7}{15}\ln\abs{3x-1}
        +\frac{2}{5}\ln\abs*{x^{2}+1}+\frac{3}{5}\arctan x+C\\
        =\,&-\frac{7}{15}\ln\abs{3x-1}
        +\frac{2}{5}\ln\brk*{x^{2}+1}+\frac{3}{5}\arctan x+C
    \end{align*}

    $\therefore$ Diperoleh $\ds\int \frac{x^{2}+x-2}{3x^{3}-x^{2}+3x-1}\,dx
    =-\frac{7}{15}\ln\abs{3x-1}
        +\frac{2}{5}\ln\abs{x^{2}+1}+\frac{3}{5}\arctan x+C$
\end{enumerate}
\end{enumerate}

\begin{center}
    \line(1,0){300}
\end{center}