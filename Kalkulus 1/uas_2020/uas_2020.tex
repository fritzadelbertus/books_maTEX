\newpage
\begin{flushright}
    \section*{\Large{Ujian Akhir Semester}}
    \addcontentsline{toc}{section}{Ujian Akhir Semester (UAS)}
    \subsection*{Tahun 2020}
    \addcontentsline{toc}{subsection}{UAS - 2020}
\end{flushright}
\vspace{0.5cm}
\hrule height 2pt
\vspace{0.5cm}
\begin{center}
    \textbf{\large{MATERI}}
    \begin{enumerate}[leftmargin=*, label={\arabic*}.]
        \item Teknik-teknik Integrasi
        \item Sketsa grafik fungsi dan kurva.
        \item Aplikasi integral dalam mencari volume benda putar.
        \item Fungsi Logaritma dan Eksponensial
        \item Fungsi Invers
    \end{enumerate}
\end{center}
\vspace{0.2cm}
\hrule height 1pt
\vspace{0.5cm}
\begin{center}
    \textbf{\large{SOAL}}
\end{center}
\begin{enumerate}[leftmargin=*, label={\arabic*}.]
\item Tentukanlah
\begin{enumerate}[label={\alph*}.]
    \item $\ds\int \frac{x\sin\sqrt{x^{2}+4}}{\sqrt{x^{2}+4}}\, dx$
    \item $\ds\int_{-\pi/2}^{\pi/2} \cos\theta\cos\brk*{\pi\sin\theta}\,d\theta$
\end{enumerate}
\item Daerah $R$ yang dibatasi oleh $y=3x^{2}-2$, $y=x^{2}$, dan $x\geq 0$, diputar 
mengelilingi sumbu-$y$.
\begin{enumerate}[label={\alph*}.]
    \item Sketsalah daerah $R$.
    \item Metode apa sajakah yang dapat digunakan untuk menghitung volume benda putar 
    yang terbentuk?
    \item Hitunglah volume benda putar tersebut dengan menggunakan salah satu metode yang 
    Anda sebutkan pada bagian b!
\end{enumerate}
\item Diberikan fungsi bernilai real $f$ dan $g$ sebagai berikut:
\[
f(x)=3+\sqrt{\frac{1}{x-1}}\,\text{ dengan }x > 1\,\text{dan }g(x)=
\int_{x}^{0}\sqrt{\frac{1}{4}+\sin^{2}t}\,dt.
\]
\begin{enumerate}[label={\alph*}.]
    \item Tentukanlah $f^{-1}(x)$, kemudian carilah $(f^{-1})'(x)$.
    \item Jika $p=g(a)$, $0\leq a \leq \frac{\pi}{2}$, tentukanlah nilai $a$ 
    yang memenuhi persamaan:
    \[
    (f^{-1})'(4)=(g^{-1})'(p).
    \]
\end{enumerate}
\item Hitunglah
\begin{enumerate}[label={\alph*}.]
    \item Jika $f(x)=\sin^{2}x+2^{\sin x}$, maka $2e^{f'(\pi)}=\dots$.
    \item $\int_{0}^{1} 10^{3x}+10^{-3x}\,dx$.
\end{enumerate}
\item Tentukanlah
\begin{enumerate}[label={\alph*}.]
    \item $\int \arctan(5x)\,dx$.
    \item $\ds \int \frac{1}{\brk*{x^{2}+4}^{3/2}}\,dx$ 
    $\ds\brk*{\text{petunjuk: } \sin(\arctan x) = \frac{x}{\sqrt{1+x^{2}}}}$.
\end{enumerate}
\end{enumerate}
\vspace{0.2cm}
\hrule height 1pt
\vspace{0.5cm}
\begin{center}
    \textbf{\large{PEMBAHASAN}}
\end{center}
\begin{enumerate}[leftmargin=*, label={\arabic*}.]
\item Halo
\end{enumerate}

\begin{center}
    \line(1,0){300}
\end{center}