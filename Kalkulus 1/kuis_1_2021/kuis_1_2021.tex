\begin{flushright}
    \textbf{\Large{Kuis 1}}
    \subsection*{Tahun 2021}
    \addcontentsline{toc}{subsection}{Kuis 1 - 2021}
\end{flushright}
\vspace{0.5cm}
\hrule height 2pt
\vspace{0.5cm}
\begin{center}
    \textbf{\large{MATERI}}
    \begin{enumerate}[leftmargin=*, label={\arabic*}.]
        \item Menyelesaikan pertidaksamaan yang melibatkan nilai mutlak.
        \item Fungsi komposisi dan menentukan domainnya.
        \item Mencari nilai limit kiri dan nilai limit kanan fungsi.
        \item Menentukan kekontinuan fungsi pada suatu titik.
    \end{enumerate}
\end{center}
\vspace{0.2cm}
\hrule height 1pt
\vspace{0.5cm}
\begin{center}
    \textbf{\large{SOAL}}
\end{center}
\begin{enumerate}[leftmargin=*, label={\arabic*}.]
\item Diketahui 
$f(x) = \abs{x-2}$ dan 
$g(x) = x+1$.
\begin{enumerate}[label={\alph*}.]
    \item Tentukanlah nilai $x$ yang memenuhi $f(x) \leq g(x)$.
    \item Jika 
    $h(x) = \sqrt{g(x)}$, maka tentukanlah 
    $(f \circ h)(x)$ dan domain dari
    $(f \circ h)(x)$. Jelaskanlah jawaban anda!
\end{enumerate}
\item Misalkan
\[
    f(x) = 
    \begin{cases}
        2-x^{2}, &\text{jika $0 \leq x < 1$}\\
        \frac{5}{2}, &\text{jika $x = 1$}\\
        \abs{2-x}, &\text{jika $1 < x \leq 3$}\\
        \frac{a}{2x-3}, &\text{jika $3 < x \leq 5$}
    \end{cases}
\]
\begin{enumerate}[label={\alph*}.]
    \item Apakah fungsi $f$ kontinu di $x=1$? Jelaskanlah jawaban anda!
    \item Tentukan nilai $a$ sehingga $f$ kontinu di $x=3$. Jelaskanlah!
\end{enumerate}
\end{enumerate}
\vspace{0.2cm}
\hrule height 1pt
\vspace{0.5cm}

\begin{center}
    \textbf{\large{PEMBAHASAN}}
\end{center}
\begin{enumerate}[leftmargin=*, label={\arabic*}.]
\item
\begin{enumerate}[label={\alph*}.]
\item Akan dicari nilai $x$ yang memenuhi
$f(x) \leq g(x) \iff \abs{x-2} \leq x+1$.\\
Hanya ada 3 aturan dasar yang diperlukan untuk menyelesaikan pertidaksamaan.
\begin{enumerate}[label={\arabic*})]
    \item Jumlahkan kedua ruas dengan bilangan yang sama.
    \item Kalikan kedua ruas dengan bilangan positif yang sama.
    \item Kalikan kedua ruas dengan bilangan negatif yang sama dan 
    mengubah arah pertidaksamaannya.
\end{enumerate}
Pada soal ini ruas kiri melibatkan nilai mutlak. Untuk menyelesaikannya 
pertidaksamaan perlu diubah kebentuk yang tidak melibatkan nilai mutlak. 
Cara yang selalu bisa digunakan adalah dengan membagi kasus pada nilai $x$.
    
Perhatikan bahwa sesuai definisi nilai mutlak
\[
\abs{x-2} = 
\begin{cases}
    x-2, &\text{jika $x-2 \geq 0$ atau $x \geq 2$}\\
    -(x-2), &\text{jika $x-2 < 0$ atau $x < 2$}
\end{cases}
\]
Sehingga pertidaksamaan ini diselesaikan dengan membagi menjadi 2 kasus 
seperti yang terlihat pada garis bilangan di bawah ini.
    
\vspace{0.2cm}
\begin{tikzpicture}
\draw[stealth-stealth] (-6,0) node[below]{$-\infty$}--(5,0) node[below]{$\infty$};
\draw (0,.1)--(0,-.1) node[below=0.2em]{$2$};
\node at (-7,.4) {$\abs{x-2}=$};
    
\node at (-3,.4) {$-(x-2)$};
\node at (3,.4) {$x-2$};
\end{tikzpicture}
    
\textbf{Kasus 1: $x < 2$}\\
Maka
\begin{align*}
    \abs{x-2} \leq x+1
    \iff &-(x-2) \leq x+1 
    &\text{definisi nilai mutlak} \\
    \iff &2-x \leq x+1
    &\text{penyederhanaan}\\
    \iff &-2x \leq -1
    &\text{kedua ruas jumlahkan $-2-x$}\\
    \iff &x \geq \frac{1}{2}
    &\text{kedua ruas kalikan $-\frac{1}{2}$}
\end{align*}
Sehingga untuk kasus $x < 2$ nilai $x$ yang memenuhi pertidaksamaan adalah 
$x \geq \frac{1}{2}$. Dengan kata lain, nilai $-\frac{1}{2} \leq x < 2$ 
memenuhi pertidaksamaan.
    
\textbf{Kasus 2: $x \geq 2$}\\
Maka
\begin{align*}
    \abs{x-2} \leq x+1
    \iff &x-2 \leq x+1 
    &\text{definisi nilai mutlak} \\
    \iff &-2 \leq 1
    &\text{kedua ruas jumlahkan $-x$}
\end{align*}
Ini menandakan untuk semua nilai $x \geq 2$ pertidaksamaan akan ujungnya 
berbentuk $-2 \leq 1$ yang selalu bernilai benar. Dengan kata lain, semua 
nilai $x \geq 2$ memenuhi pertidaksamaan.
    
Semua nilai $x$ yang memenuhi $f(x) \leq g(x)$ adalah gabungan nilai $x$ 
dari kedua kasus. Sehingga nilai $x$ yang memenuhi adalah 
$\set*{\frac{1}{2} \leq x < 2 \cup x \geq 2}$ atau 
$\set*{x \in \mathbb{R} \mid x \geq \frac{1}{2}}$
atau $\cio*{\frac{1}{2}, \infty}$
    
$\therefore$ Nilai $x$ yang memenuhi $f(x) \leq g(x)$ adalah
$\set*{x \in \mathbb{R} \mid x \geq \frac{1}{2}}$ atau
$\cio*{\frac{1}{2}, \infty}$.
    
\vspace{0.1cm}
\textbf{Catatan:}\\
Salah satu cara untuk mengubah pertidaksamaan yang melibatkan nilai mutlak 
ke pertidaksamaan yang tidak adalah dengan menguadratkan kedua ruas. Soal 
ini \textbf{TIDAK} dapat diselesaikan dengan cara tersebut. Hal ini 
dikarenakan cara tersebut memiliki syarat yaitu kedua ruasnya harus bernilai 
positif. Pada kasus ini ruas kiri dapat bernilai negatif.
\begin{center}
    \line(1,0){150}
\end{center}
\item Akan dicari $(f \circ h)(x)$ dan domainnya.\\
Soal memberikan informasi bahwa $f(x) = \abs{x-2}$, $g(x) = x+1$ dan 
$h(x) = \sqrt{g(x)} = \sqrt{x+1}$.\\
Dengan definisi fungsi komposisi, maka
\[
    (f \circ h)(x) = f(h(x)) = f\brk*{\sqrt{x+1}} = \abs*{\sqrt{x+1}-2}
\]
Domain dari komposisi $f \circ h$ adalah himpunan nilai $x$ dimana
\begin{enumerate}[label={\arabic*})]
    \item $x$ berada di dalam domain dari $h$
    \item $h(x)$ berada di dalam domain dari $f$
\end{enumerate}
Karena $h(x) = \sqrt{x+1}$ memiliki bentuk akar maka domain natural dari 
fungsi $h(x)$ adalah \\ $x+1 \geq 0 \iff x \geq -1$.
Selanjutnya karena range dari $h$ atau semua nilai $h(x)$ positif, maka
semua $h(x)$ berada di dalam domain dari $f$. Sehingga domain dari 
$f \circ h$ adalah $\set*{x \geq -1}$ atau $\cio{-1,\infty}$
    
$\therefore$ $(f \circ h)(x) = \abs*{\sqrt{x+1} - 2}$ dengan domain 
$\set*{x \geq -1}$ atau $\cio{-1,\infty}$ 
    
\vspace{0.1cm}
\textbf{Catatan:}\\
Cara mudah untuk mencari domainnya adalah mencari langsung domain 
$f \circ h$ dari $(f \circ h)(x)$ dan mengambil irisannya dengan 
domain dari $h$. (Lihat Kuis 1 2022 - 1b)
\begin{center}
    \line(1,0){300}
\end{center}
\end{enumerate}
\item
\begin{enumerate}[label={\alph*}.]
\item Akan dibuktikan $f$ tidak kontinu di $x=1$.\\
Tiga syarat untuk membuktikan $f$ kontinu di $x=1$ adalah
\begin{enumerate}[label={\arabic*}.]
    \item $\lim_{x\to 1} f(x)$ ada.
    \item $f(1)$ ada.
    \item $\lim_{x\to 1} f(x) = f(1)$
\end{enumerate}
Membuktikan tidak berlaku maka menunjukan salah satu syarat tersebut tidak
terpenuhi.\\
$\lim_{x\to 1} f(x)$ ada karena limit kanan dan limit kiri ada dan bernilai sama.
\begin{align*}
    \lim_{x\to 1^{-}} f(x) 
    &= \lim_{x\to 1^{-}} 2-x^{2}
    &\text{definisi fungsi $f$}\\
    &= 2-1^{2} = 1
    &\text{teorema subtitusi}\\
    \lim_{x\to 1^{+}} f(x) 
    &= \lim_{x\to 1^{+}} \abs{2-x}
    &\text{definisi fungsi $f$}\\
    &= \abs{2-1} = 1
    &\text{teorema subtitusi}
\end{align*}
$f(1)$ ada karena sesuai definisi soal bahwa $f(1) = \frac{5}{2}$.\\
Syarat ketiga tidak terpenuhi karena
\[
    \lim_{x\to 1} f(x) = 1 \neq \frac{5}{2} = f(1)
\].
Sehingga $f$ tidak kontinu di $x=1$ karena tidak memenuhi syarat ketiga

$\therefore$ $f$ tidak kontinu di $x=1$ karena $\lim_{x\to 1} f(x) \neq f(1)$
\begin{center}
    \line(1,0){150}
\end{center}
\item Akan dicari nilai $a$ sehingga $f$ kontinu di $x=3$.\\
Sama seperti sebelumnya, tiga syarat untuk membuktikan $f$ kontinu di
$x=3$ adalah:
\begin{enumerate}[label={\arabic*}.]
    \item $\lim_{x\to 3} f(x)$ ada.
    \item $f(3)$ ada.
    \item $\lim_{x\to 3} f(x) = f(3)$
\end{enumerate}
Mulai dengan menentukan $a$ sehingga limit fungsi di $x=3$ ada. Hal ini 
dapat dilakukan dengan menyamakan limit kanan dan limit kiri.
\begin{align*}
    \lim_{x\to 3^{-}} f(x) 
    &= \lim_{x\to 3^{-}} \abs{2-x}
    &\text{definisi fungsi $f$}\\
    &= \abs{2-3} = 1
    &\text{teorema subtitusi}\\
    \lim_{x\to 3^{+}} f(x) 
    &= \lim_{x\to 3^{+}} \frac{a}{2x-3}
    &\text{definisi fungsi $f$}\\
    &= \frac{a}{6-3} = \frac{a}{3}
    &\text{teorema subtitusi}
\end{align*}
Pilih $a = 3$ sehingga
\[
    \lim_{x\to 3^{+}} f(x) = \frac{a}{3} = \frac{3}{3} = 1 = \lim_{x\to 3^{-}} f(x) 
\]
Syarat pertama terpenuhi yaitu $\lim_{x\to 3} f(x) = 1$ ada.\\
Syarat kedua terpenuhi karena dari definisi fungsi $f(3) = \abs{2-3} = 1$.\\
Syarat ketiga juga terpenuhi karena $\lim_{x\to 3} f(x) = 1 = f(3)$.
Sehingga dengan memilih $a=3$ fungsi $f$ kontinu di $x=3$.

$\therefore$ Nilai $a$ sehingga $f$ kontinu di $x=3$ adalah $a=3$.
\end{enumerate}
\end{enumerate}
\begin{center}
    \line(1,0){300}
\end{center}