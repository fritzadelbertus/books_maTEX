\begin{flushright}
    \textbf{\Large{Kuis 2}}
    \subsection*{Tahun 2021}
    \addcontentsline{toc}{subsection}{Kuis 2 - 2021}
\end{flushright}
\vspace{0.5cm}
\hrule height 2pt
\vspace{0.5cm}
\begin{center}
    \textbf{\large{MATERI}}
    \begin{enumerate}[leftmargin=*, label={\arabic*}.]
        \item Sketsa grafik fungsi dan kurva.
        \item Mencari nilai integral.
        \item Aplikasi integral dalam mencari luas daerah.
        \item Aplikasi integral dalam mencari volume benda putar.
        \item Teorema Dasar Kalkulus
    \end{enumerate}
\end{center}
\vspace{0.2cm}
\hrule height 1pt
\vspace{0.5cm}
\begin{center}
    \textbf{\large{SOAL}}
\end{center}
\begin{enumerate}[leftmargin=*, label={\arabic*}.]
\item Misalkan $D$ adalah suatu daerah yang dibatasi di kiri oleh garis $x+y=2$, 
di kanan oleh kurva $y=x^{2}$ dan diatas oleh garis $y=2$.
\begin{enumerate}[label={\alph*}.]
    \item Gambarkanlah daerah $D$.
    \item Tentukanlah luas daerah $D$.
\end{enumerate}
\item Misalkan $A$ adalah suatu daerah yang dibatasi oleh $y=x^{2}$ dan $y=4x-x^{2}$.
\begin{enumerate}[label={\alph*}.]
    \item Gambarkanlah daerah $A$.
    \item Tentukan volume benda putar dari daerah $A$ yang diputar mengelilingi $x=5$.
\end{enumerate}
\item Tentukanlah turunan dari $y$ berikut terhadap $x$
\[
y = \int_{0}^{\ln x}\sin e^{t}\,dt
\]
\end{enumerate}
\vspace{0.2cm}
\hrule height 1pt
\vspace{0.5cm}

\begin{center}
    \textbf{\large{PEMBAHASAN}}
\end{center}
\begin{enumerate}[leftmargin=*, label={\arabic*}.]
\item Diberikan $D$ adalah suatu daerah yang dibatasi di kiri oleh garis $x+y=2$, 
di kanan oleh kurva $y=x^{2}$ dan diatas oleh garis $y=2$.
\begin{enumerate}[label={\alph*}.]
    \item Akan digambarkan daerah $D$.\\
    Uji dua titik untuk sketsa $y=x+2$ dan tiga titik untuk $y=x^2$.
\begin{center}
    \begin{tabular}{|c|c|c|}\hline
        $x$  & $0$ & $1$ \\ \hline
        $y=2-x$ & $2$ & $1$ \\ \hline
    \end{tabular}\quad
    \begin{tabular}{|c|c|c|c|}\hline
        $x$ & $-1$ & $0$ & $1$ \\ \hline
        $y=x^{2}$ & $1$ & $0$ & $1$ \\ \hline
    \end{tabular}
\end{center}
    Diperoleh daerah yang dibatasi seperti berikut:
\begin{center}
    \begin{tikzpicture}[>=stealth]
    \begin{axis}[
        xmin=-3,xmax=3,
        ymin=-0.5,ymax=5,
        axis x line=middle,
        axis y line=middle,
        axis line style=<->,
        xlabel={$x$},
        ylabel={$y$},
        ]
        \addplot [name path=f, no marks,blue, domain=-3:3,samples=50]({x},{2-x});
        \node [blue] at (2,0.7){\scalebox{0.7}{$x+y=2$}};
        \addplot [name path=g, no marks,red, domain=-3:3,samples=50]({x},{x^2});
        \node [red] at (1.2,3){\scalebox{0.7}{$y=x^{2}$}};
        \addplot [name path=h, no marks,black, domain=-3:3,samples=50]({x},{2});
        \node [black] at (-2,2.2){\scalebox{0.7}{$y=2$}};
        \node [draw, shape = circle, fill = blue, minimum size = 0.1cm, inner sep=0pt] at (0,2){};
        \node [draw, shape = circle, fill = purple, minimum size = 0.1cm, inner sep=0pt] at (1,1){};
        \node [draw, shape = circle, fill = red, minimum size = 0.1cm, inner sep=0pt] at (0,0){};
        \node [draw, shape = circle, fill = red, minimum size = 0.1cm, inner sep=0pt] at (-1,1){};
        \fill[thick, cyan,opacity=0.5] plot[domain=0:1] (\x,2-\x) -- plot[domain=1:1.414] ({\x},{\x^2});
        \node [black] at (0.8,1.7){$D$};
    \end{axis}
    \end{tikzpicture}
\end{center}
$\therefore$ Telah digambarkan daerah $D$.
\begin{center}
    \line(1,0){150}
\end{center}
    \item Akan dicari luas daerah $D$.\\
    Gunakan integral, partisi daerah tegak lurus dengan sumbu-$x$. Partisi akan terbagi dua, partisi 
    yang terbatas di atas oleh $y=2$ dan di bawah oleh $x+y=2$ dan partisi yang terbatas di atas oleh 
    $y=2$ dan di bawah oleh $y=x^2$. Partisi pertama bermulai dari $x=0$ sampai titik potong $x+y=2$ 
    dengan $y=x^{2}$ dan partisi kedua dimulai dari titik potong sebelumnya ke titik potong $y=x^{2}$ 
    dan $y=2$.

    Akan dicari titik potong $x+y=2$ dengan $y=x^{2}$.
    \begin{align*}
        x+y=2,\,y=x^{2} &\Longrightarrow 2-x = x^{2}\\
        &\iff x^{2}+x-2=0\\
        &\iff (x+2)(x-1)=0
    \end{align*}
    Diperoleh $x=-2$ dan $x=1$ pilih $x=1$ karena $x=-2$ tidak membatasi daerah $D$.

    Akan dicari titik potong $y=2$ dengan $y=x^{2}$.
    \begin{align*}
        y=2,\,y=x^{2} &\Longrightarrow 2 = x^{2}\\
        &\iff x^2-2=0\\
    \end{align*}
    Diperoleh $x=-\sqrt{2}$ dan $x=\sqrt{2}$ pilih $x=\sqrt{2}$ karena $x=-\sqrt{2}$ 
    tidak membatasi daerah $D$.

    Sekarang integralkan untuk memperoleh luas daerah $D$.
    \begin{align*}
        \text{Luas $D$} &= \int_{0}^{1} (2)-(2-x)\,dx + \int_{1}^{\sqrt{2}} (2)-(x^{2})\,dx\\
        &= \int_{0}^{1} x\,dx + \int_{1}^{\sqrt{2}} 2-x^{2}\,dx\\
        &= \eval{\frac{1}{2}x^{2}}{0}{1}+\eval{2x-\frac{1}{3}x^{3}}{1}{\sqrt{2}}\\
        &= \brk*{\frac{1}{2}(1)^{3}-\frac{1}{2}(0)^{3}}
        +\brk*{\brk*{2\brk*{\sqrt{2}}-\frac{1}{3}\brk*{\sqrt{2}}^{3}}
        -\brk*{2(1)-\frac{1}{3}(1)^{3}}}\\
        &=\frac{1}{2}+\brk*{\frac{4\sqrt{2}}{3}-\frac{5}{3}}=\frac{8\sqrt{2}-7}{6}
    \end{align*}

    $\therefore$ Diperoleh luas daerah $D$ adalah $\ds \frac{8\sqrt{2}-7}{6}$
\end{enumerate}
\begin{center}
    \line(1,0){300}
\end{center}
\item Diberikan $A$ adalah suatu daerah yang dibatasi oleh $y=x^{2}$ dan $y=4x-x^{2}$.
\begin{enumerate}[label={\alph*}.]
    \item Akan digambar daerah $A$.\\
    Gunakan tiga titik dari masing-masing persamaan untuk mensketsa kurva.
    \begin{center}
        \begin{tabular}{|c|c|c|c|}\hline
            $x$ & $-1$ & $0$ & $1$ \\ \hline
            $y=x^{2}$ & $1$ & $0$ & $1$ \\ \hline
        \end{tabular}\quad
        \begin{tabular}{|c|c|c|c|}\hline
            $x$ & $0$ & $1$ & $2$ \\ \hline
            $y=4x-x^{2}$ & $0$ & $3$ & $4$ \\ \hline
        \end{tabular}
    \end{center}
    Diperoleh daerah yang dibatasi seperti berikut:
\begin{center}
    \begin{tikzpicture}[>=stealth]
    \begin{axis}[
        xmin=-1.5,xmax=6,
        ymin=-0.5,ymax=5,
        axis x line=middle,
        axis y line=middle,
        axis line style=<->,
        xlabel={$x$},
        ylabel={$y$},
        ]
        \addplot [name path=f, no marks,red, domain=-1.5:5.5,samples=50]({x},{4*x-x^2});
        \node [red] at (2.8,0.7){\scalebox{0.7}{$y=4x-x^{2}$}};
        \addplot [name path=g, no marks,blue, domain=-1.5:5.5,samples=50]({x},{x^2});
        \node [blue] at (1.5,4.5){\scalebox{0.7}{$y=x^{2}$}};
        \addplot [name path=h, no marks,black, domain=-1:5,samples=50]({5},{x});
        \node [black] at (4.4,4.5){\scalebox{0.7}{$x=5$}};
        \node [draw, shape = circle, fill = blue, minimum size = 0.1cm, inner sep=0pt] at (-1,1){};
        \node [draw, shape = circle, fill = blue, minimum size = 0.1cm, inner sep=0pt] at (1,1){};
        \node [draw, shape = circle, fill = purple, minimum size = 0.1cm, inner sep=0pt] at (0,0){};
        \node [draw, shape = circle, fill = red, minimum size = 0.1cm, inner sep=0pt] at (1,3){};
        \node [draw, shape = circle, fill = red, minimum size = 0.1cm, inner sep=0pt] at (2,4){};
        \addplot [thick, color=cyan, fill=cyan, fill opacity=0.5]
        fill between[of=f and g, soft clip={domain=-0:2},]; 
        \node [black] at (1,2){$A$};
    \end{axis}
    \end{tikzpicture}
\end{center}
$\therefore$ Telah digambarkan daerah $A$.
\begin{center}
    \line(1,0){150}
\end{center}
    \item Akan dicari volume benda putar dari daerah $A$ yang diputar mengelilingi $x=5$.\\
    Gunakan metode kulit tabung, maka partisi akan sejajar dengan garis $x=5$. Tinggi 
    tabung dibatas diatas oleh kurva $y=4x-x^{2}$ dan dibawah oleh $y=x^{2}$. Sebelum menentukan 
    jari-jari kulit tabung akan ditentukan titik potong kedua kurva.
    \begin{align*}
        y = x^{2},\,y=4x-x^{2} &\Longrightarrow x^{2}=4x-x^{2}\\
        & \iff 2x^{2}-4x=0\\
        & \iff (2x)(x-2)=0
    \end{align*}
    Sehingga $x=0$ dan $x=2$ adalah titik potong kedua kurva. $x=0$ menjadi ujung terjauh 
    dari garis putar, sedangkan $x=2$ menjadi ujung terdekat dari garis putar. 
    Rumus jari-jari kulit tabung adalah $5-x$.

    Gunakan integral untuk menghitung volume benda putar.
    \begin{align*}
        V &= \int \text{Luas Kulit Tabung}\\
        &=\int_{0}^{2} 2\pi(\text{jari-jari})(\text{tinggi})\,dx\\
        &=2\pi\int_{0}^{2} (5-x)\brk*{(4x-x^{2})-(x^{2})}\,dx\\
        &=2\pi\int_{0}^{2} (5-x)\brk*{4x-2x^{2}}\,dx\\
        &=2\pi\int_{0}^{2} \brk*{2x^{3}-14x^{2}+20x}\,dx\\
        &=2\pi\eval{\frac{1}{2}x^{4}-\frac{14}{3}x^{3}+10x^{2}}{0}{2}\\
        &=2\pi\brk*{\brk*{\frac{1}{2}(2)^{4}-\frac{14}{3}(2)^{3}+10(2)^{2}}-(0)}\\
        &=2\pi\brk*{8+40-\frac{112}{3}}=\frac{64\pi}{3}
    \end{align*}
    Berikut ilustrasi hasil benda padat
\begin{center}
\begin{tikzpicture}
    \begin{axis}[view={-30}{30},colormap/Spectral-9]
    \begin{scope}
        \def\rx{(5)}
        \def\fx{(x^2)}
        \addplot3[surf,shader=flat,
            samples=20,
            domain=0:2,y domain=0:2*pi,
            z buffer=sort]
            ({(((x-\rx) * cos(deg(y)))+\rx)}, {(((x-\rx) * sin(deg(y)))+\rx)}, {\fx});
    \end{scope}
    \begin{scope}
        \def\rx{(5)}
        \def\fx{(4*x-x^2)}
        \addplot3[surf,shader=flat,
            samples=20,
            domain=0:2,y domain=0:2*pi,
            z buffer=sort]
            ({(((x-\rx) * cos(deg(y)))+\rx)}, {(((x-\rx) * sin(deg(y)))+\rx)}, {\fx});
    \end{scope}
    \addplot3[black,shader=flat,
            samples=25,
            domain=0:4,y domain=0:2*pi,
            z buffer=sort]
            (5,5,x);
    \node at (4,3.7,5) {$x=5$};
    \end{axis}
\end{tikzpicture}       
\end{center}
$\therefore$ Diperoleh volume benda putar dari daerah $A$ yang diputar mengelilingi $x=5$ 
adalah $\ds\frac{64\pi}{3}$.
\end{enumerate}
\begin{center}
    \line(1,0){300}
\end{center}
\item Akan dicari turunan dari $y$ berikut terhadap $x$
\[
y = \int_{0}^{\ln x}\sin e^{t}\,dt
\]

Gunakan aturan rantai dan Teorema Dasar Kalkulus.
\begin{align*}
    \frac{dy}{dx} &= \drv{x}{\int_{0}^{\ln x}\sin e^{t}\,dt}\\
    &= \drv{(\ln x)}{\int_{0}^{\ln x}\sin e^{t}\,dt}\frac{d(\ln x)}{dx}
    &\text{aturan rantai}\\
    &=\brk*{\sin e^{\ln x}}\brk*{\frac{1}{x}}
    &\text{Teorema Dasar Kalkulus}\\
    &=\frac{\sin x}{x}
    &\text{karena }e^{\ln x} = x
\end{align*}

$\therefore$ Diperoleh turunan pertama dari $y$ terhadap $x$ adalah $\ds\frac{\sin x}{x}$.

\end{enumerate}
\begin{center}
    \line(1,0){300}
\end{center}