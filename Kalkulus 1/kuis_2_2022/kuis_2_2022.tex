\begin{flushright}
    \textbf{\Large{Kuis 2}}
    \subsection*{Tahun 2022}
    \addcontentsline{toc}{subsection}{Kuis 2 - 2022}
\end{flushright}
\vspace{0.5cm}
\hrule height 2pt
\vspace{0.5cm}
\begin{center}
    \textbf{\large{MATERI}}
    \begin{enumerate}[leftmargin=*, label={\arabic*}.]
        \item Sketsa grafik fungsi dan kurva.
        \item Mencari nilai integral.
        \item Mencari nilai integral dengan metode subtitusi.
        \item Aplikasi integral dalam mencari luas daerah.
        \item Aplikasi integral dalam mencari volume benda putar.
        \item Fungsi Transendental
    \end{enumerate}
\end{center}
\vspace{0.2cm}
\hrule height 1pt
\vspace{0.5cm}
\begin{center}
    \textbf{\large{SOAL}}
\end{center}
\begin{enumerate}[leftmargin=*, label={\arabic*}.]
\item Misalkan $R$ adalah daerah yang dibatasi oleh kurva $y=x-x^{2}$ 
dengan sumbu-$x$.
\begin{enumerate}[label={\alph*}.]
    \item Buatlah sketsa daerah $R$.
    \item Hitunglah luas daerah $R$.
    \item Hitung volume benda padat yang terbentuk bila $R$ diputar 
    mengelilingi garis $y=-1$.
\end{enumerate}
\item Tentukanlah $\ds \frac{dy}{dx}$ dari $y=\brk*{\ln x^{2}}^{2x+3}$.
\item Hitunglah $\int e^{x+e^{x}}\,dx$.
\end{enumerate}
\vspace{0.2cm}
\hrule height 1pt
\vspace{0.5cm}

\begin{center}
    \textbf{\large{PEMBAHASAN}}
\end{center}
\begin{enumerate}[leftmargin=*, label={\arabic*}.]
\item Diberikan $R$ adalah daerah yang dibatasi oleh kurva $y=x-x^{2}$ 
dengan sumbu-$x$.
\begin{enumerate}[label={\alph*}.]
    \item Akan dibuat sketsa daerah $R$.\\
    Gunakan tiga titik dari $y=x-x^{2}$ untuk mensketsa kurva.
\begin{center}
    \begin{tabular}{|c|c|c|c|}\hline
        $x$ & $0$ & $1$ & $2$ \\ \hline
        $y=x-x^{2}$ & $0$ & $0$ & $-2$ \\ \hline
    \end{tabular}
\end{center}
Diperoleh daerah yang dibatasi seperti berikut:
\begin{center}
    \begin{tikzpicture}[>=stealth]
    \begin{axis}[
        xmin=-2,xmax=3,
        ymin=-2.5,ymax=0.5,
        axis x line=middle,
        axis y line=middle,
        axis line style=<->,
        xlabel={$x$},
        ylabel={$y$},
        ]
        \addplot [name path=f, no marks,blue, domain=-2:3,samples=50]({x},{x-x^2});
        \node [blue] at (2,-0.5){\scalebox{0.7}{$y=x-x^{2}$}};
        \addplot [name path=g, no marks,transparent, domain=-2:3,samples=50]({x},{0});
        \addplot [no marks,black, domain=-2:3,samples=50]({x},{-1});
        \node [black] at (2.2,-1.15){\scalebox{0.7}{$y=-1$}};
        \node [draw, shape = circle, fill = blue, minimum size = 0.1cm, inner sep=0pt] at (0,0){};
        \node [draw, shape = circle, fill = blue, minimum size = 0.1cm, inner sep=0pt] at (1,0){};
        \node [draw, shape = circle, fill = blue, minimum size = 0.1cm, inner sep=0pt] at (2,-2){};
        \addplot [thick, color=cyan, fill=cyan, fill opacity=0.5]
        fill between[of=f and g, soft clip={domain=0:1},]; 
        \node [black] at (0.5,0.1){$R$};
    \end{axis}
    \end{tikzpicture}
\end{center}
$\therefore$ Telah disketsa daerah $R$.
\begin{center}
    \line(1,0){150}
\end{center}
    \item Akan dicari luas daerah $R$.
    Gunakan integral, cari ujung kanan dan kiri daerah $R$.
    \begin{align*}
        y=x-x^{2}, y=0 &\Longrightarrow x-x^{2}=0\\
        &\iff x(x-1) = 0
    \end{align*}
    Diperoleh titik potong di $x=0$ dan $x=1$ yang sekaligus menjadi ujung kanan 
    dan kiri daerah $R$. Lakukan integrasi
    \begin{align*}
        \text{Luas R} = &\int_{0}^{1} x-x^{2}\,dx\\
        &=\eval{\frac{1}{2}x^{2}-\frac{1}{3}x^{3}}{0}{1}\\
        &=\brk*{\frac{1}{2}(1)^{2}-\frac{1}{3}(1)^{3}}-(0)\\
        &=\frac{1}{6}
    \end{align*}
    $\therefore$ Diperoleh luas daerah $R$ adalah $\ds\frac{1}{6}$.
\begin{center}
    \line(1,0){150}
\end{center}
    \item Akan dicari volume benda padat yang terbentuk bila $R$ diputar 
    mengelilingi garis $y=-1$.\\
    Gunakan metode cincin. Dari hasil sketsa soal sebelumnya terlihat, Jari-jari luar 
    cincin adalah jarak dari $y=-1$ ke kurva $y=x-x^{2}$ yaitu $x-x^{2}+1$, dan 
    jari-jari dalam cincin adalah jarak dari $y=-1$ ke sumbu-$x$ yaitu $1$. Dari hasil 
    sebelumnya sudah diperoleh batas integrasi dari $x=0$ sampai $x=1$.

    Gunakan integral untuk memperoleh volume
    \begin{align*}
        V &= \int \text{Luas bagian luar} - \text{Luas bagian dalam}\\
        &=\int_{0}^{1} \pi\brk*{x-x^{2}+1}^2 - \pi(1)^2\,dx\\
        &=\pi \int_{0}^{1} \brk*{x^{4}-2x^{3}-x^{2}+2x+1}-1\,dx\\
        &=\pi \int_{0}^{1} x^{4}-2x^{3}-x^{2}+2x\,dx\\
        &=\pi\eval{\frac{1}{5}x^{5}-\frac{1}{2}x^{4}-\frac{1}{3}x^{3}+x^2}{0}{1}\\
        &=\pi\brk*{\brk*{\frac{1}{5}(1)^{5}-\frac{1}{2}(1)^{4}-\frac{1}{3}(1)^{3}+(1)^2}-(0)}\\
        &=\frac{11}{30}\pi
    \end{align*}
    Berikut ilustrasi hasil benda padat
\begin{center}
\begin{tikzpicture}
    \begin{axis}[view={-30}{30},colormap/Spectral-9]
    \begin{scope}
        \def\ry{(-1)}
        \def\fx{(x-x^2)}
        \addplot3[surf,shader=flat,
            samples=25,
            domain=0:1,y domain=0:2*pi,
            z buffer=sort]
            (x,{((\fx-\ry) * cos(deg(y))+\ry)}, {((\fx-\ry) * sin(deg(y))+\ry)});
    \end{scope}
    \begin{scope}
        \def\ry{(-1)}
        \def\fx{(0)}
        \addplot3[surf,shader=flat,
            samples=25,
            domain=0:1,y domain=0:2*pi,
            z buffer=sort]
            (x,{((\fx-\ry) * cos(deg(y))+\ry)}, {((\fx-\ry) * sin(deg(y))+\ry)});
    \end{scope}
    \addplot3[black,shader=flat,
            samples=25,
            domain=0:1,
            z buffer=sort]
            (x,-1,-1);
    \node at (0,-0.5,-1) {$y=-1$};
    \end{axis}
\end{tikzpicture}       
\end{center}
$\therefore$ Diperoleh volume benda padat bila $R$ diputar 
mengelilingi garis $y=-1$ adalah $\ds \frac{11}{30}\pi$.
\end{enumerate}
\begin{center}
    \line(1,0){300}
\end{center}
\item Akan dicari $\ds \frac{dy}{dx}$ dari $y=\brk*{\ln x^{2}}^{2x+3}$.\\
Masukan kedua ruas ke bentuk logaritma natural dan turunkan.
\begin{align*}
    &\ln(y) = \ln\brk*{\brk*{\ln x^{2}}^{2x+3}}\\
    \iff &\ln(y) = (2x+3)\ln\brk*{\ln x^{2}}\\
    &\drv{x}{\ln(y)} = \drv{x}{(2x+3)\ln\brk*{\ln x^{2}}}\\
    \iff &\frac{1}{y}\frac{dy}{dx} = 
    \drv{x}{2x+3}\ln\brk*{\ln x^{2}}+(2x+3)\drv{x}{\ln\brk*{\ln x^{2}}}\\
    \iff &\frac{dy}{dx} = y\brk*{2\ln\brk*{\ln x^{2}}
    +(2x+3)\frac{1}{\ln x^{2}}\drv{x}{\ln x^{2}}}\\
    \iff &\frac{dy}{dx} = y\brk*{2\ln\brk*{\ln x^{2}}
    +(2x+3)\frac{1}{\ln x^{2}}\frac{1}{x^{2}}\drv{x}{x^{2}}}\\
    \iff &\frac{dy}{dx} = y\brk*{2\ln\brk*{\ln x^{2}}
    +(2x+3)\frac{1}{x^{2}\ln x^{2}}(2x)}\\
    \iff &\frac{dy}{dx} = 2y\brk*{\ln\brk*{\ln x^{2}}+\frac{2x+3}{x\ln x^{2}}}\\
    &\text{Subtitusi $y=\brk*{\ln x^{2}}^{2x+3}$}\\
    \iff &\frac{dy}{dx} = 2\brk*{\ln x^{2}}^{2x+3}
    \brk*{\ln\brk*{\ln x^{2}}+\frac{2x+3}{x\ln x^{2}}}
\end{align*}
$\therefore$ Diperoleh $\ds \frac{dy}{dx}$ dari $y=\brk*{\ln x^{2}}^{2x+3}$ adalah 
$\ds 2\brk*{\ln x^{2}}^{2x+3}\brk*{\ln\brk*{\ln x^{2}}+\frac{2x+3}{x\ln x^{2}}}$
\begin{center}
    \line(1,0){300}
\end{center}
\item Akan dicari $\int e^{x+e^{x}}\,dx$.\\
Gunakan metode subtitusi untuk menyelesaikan integral.

Subtitusi
    \[
        u = e^{x},\quad \frac{du}{dx} = e^{x} \iff du = e^{x}\,dx
    \]
    Sehingga 
    \begin{align*}
        \int e^{x+e^{x}}\,dx
        &= \int e^{x}e^{e^{x}}\,dx\\
        &= \int e^{e^{x}}e^{x}\,dx\\
        &= \int e^{u}\,du\\
        &= e^{u} + C &\text{subtitusi balik $u = e^{x}$}\\
        &= e^{e^{x}} + C\\
    \end{align*}
    
    $\therefore$ $\ds \int e^{x+e^{x}}\,dx = e^{e^{x}} + C$
\end{enumerate}
\begin{center}
    \line(1,0){300}
\end{center}