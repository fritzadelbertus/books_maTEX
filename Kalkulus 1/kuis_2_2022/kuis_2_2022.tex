\begin{flushright}
    \textbf{\Large{Kuis 2}}
    \subsection*{Tahun 2022}
    \addcontentsline{toc}{subsection}{Kuis 2 - 2022}
\end{flushright}
\vspace{0.5cm}
\hrule height 2pt
\vspace{0.5cm}
\begin{center}
    \textbf{\large{MATERI}}
    \begin{enumerate}[leftmargin=*, label={\arabic*}.]
        \item Menyelesaikan pertidaksamaan yang melibatkan bentuk rasional.
        \item Fungsi komposisi dan menentukan domainnya.
        \item Mencari nilai limit dari fungsi.
        \item Menentukan kekontinuan fungsi pada suatu titik.
    \end{enumerate}
\end{center}
\vspace{0.2cm}
\hrule height 1pt
\vspace{0.5cm}
\begin{center}
    \textbf{\large{SOAL}}
\end{center}
\begin{enumerate}[leftmargin=*, label={\arabic*}.]
\item Misalkan $R$ adalah daerah yang dibatasi oleh kurva $y=x-x^{2}$ 
dengan sumbu-$x$.
\begin{enumerate}[label={\alph*}.]
    \item Buatlah sketsa daerah $R$.
    \item Hitunglah luas daerah $R$.
    \item Hitung volume benda padat yang terbentuk bila $R$ diputar 
    mengelilingi garis $y=-1$.
\end{enumerate}
\item Tentukanlah $\ds \frac{dy}{dx}$ dari $y=\brk*{\ln x^{2}}^{2x+3}$.
\item Hitunglah $\int e^{x+e^{x}}\,dx$.
\end{enumerate}
\vspace{0.2cm}
\hrule height 1pt
\vspace{0.5cm}

\begin{center}
    \textbf{\large{PEMBAHASAN}}
\end{center}
\begin{enumerate}[leftmargin=*, label={\arabic*}.]
\item Halo
\end{enumerate}
