\newpage
\begin{flushright}
    \textbf{\Large{Ujian Tengah Semester}}
    \subsection*{Tahun 2022}
    \addcontentsline{toc}{subsection}{UTS - 2022}
\end{flushright}
\vspace{0.5cm}
\hrule height 2pt
\vspace{0.5cm}
\begin{center}
    \textbf{\large{MATERI}}
    \begin{enumerate}[leftmargin=*, label={\arabic*}.]
        \item Memahami konsep nilai mutlak
        \item Menyelesaikan pertidaksamaan yang melibatkan nilai mutlak.
        \item Mensketsa grafik fungsi.
        \item Fungsi komposisi dan menentukan domain dan rangenya.
        \item Mencari nilai limit kiri dan nilai limit kanan fungsi.
        \item Mencari nilai limit fungsi di tak hingga.
        \item Menentukan kekontinuan fungsi pada suatu titik.
        \item Menyelesaikan masalah turunan implisit.
        \item Menyelesaikan permasalahan maksimum dan minimum.
    \end{enumerate}
\end{center}
\vspace{0.2cm}
\hrule height 1pt
\vspace{0.5cm}
\begin{center}
    \textbf{\large{SOAL}}
\end{center}
\begin{enumerate}[leftmargin=*, label={\arabic*}.]
\item 
\begin{enumerate}[label={\alph*}.]
    \item Dua sisi paralel dari suatu persegi panjang diperpanjang dengan laju 
    $2$ cm/dt. Sementara dua sisi lainnya diperpendek, sedemikian rupa sehingga 
    bentuknya tetap merupakan persegi dengan luas konstan $A=50$ cm$^2$.
    \begin{enumerate}[label={\roman*}.]
        \item Berapa laju keliling $P$ ketika panjang salah satu sisinya sekarang 
        adalah $10$ cm?
        \item Berapa panjang sisi-sisinya sekarang ketika keliling $P$ berhenti 
        berkurang?
    \end{enumerate}
    \item Misalkan diberikan $\ds\sin^{3}(x^2+y^2)+\frac{7x+\sin y^{2}}{x-1}=5$.
    Tentukan $\ds\frac{dy}{dx}$.
\end{enumerate}
\item Diketahui $f$ adalah fungsi berikut: $f(x)=x^{4}-4x^{3}+10$. Tentukan:
\begin{enumerate}[label={\alph*}.]
    \item Interval dimana fungsi $f$ naik dan turun.
    \item Nilai-nilai maksimum dan minimum lokal fungsi $f$ (jika ada).
    \item Interval dimana fungsi $f$ cekung ke atas dan cekung ke bawah.
    \item Titik-titik belok fungsi $f$ (jika ada).
    \item Sketsa Grafik fungsi $f$.
\end{enumerate}

\end{enumerate}
\vspace{0.2cm}
\hrule height 1pt
\vspace{0.5cm}
\begin{center}
    \textbf{\large{PEMBAHASAN}}
\end{center}
\begin{enumerate}[leftmargin=*, label={\arabic*}.]
\item Halo

\end{enumerate}

\begin{center}
    \line(1,0){300}
\end{center}