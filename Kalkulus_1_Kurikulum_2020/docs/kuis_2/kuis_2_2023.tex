\begin{flushright}
    \textbf{\Large{Kuis 2}}
    \subsection*{Tahun 2023}
    \addcontentsline{toc}{subsection}{Kuis 2 - 2023}
\end{flushright}


\vspace{0.5cm}\hrule height 2pt\vspace{0.5cm}


\begin{center}
\textbf{\large{MATERI}}
\begin{enumerate}[leftmargin=*, label={\arabic*}.]
\item Sketsa grafik fungsi dan kurva.
\item Aplikasi integral dalam mencari volume benda putar.
\item Fungsi Logaritma dan Eksponensial
\item Fungsi Invers
\end{enumerate}
\end{center}


\vspace{0.2cm}\hrule height 1pt\vspace{0.5cm}


\begin{center}
\textbf{\large{SOAL}}
\end{center}
\begin{enumerate}[leftmargin=*, label={\arabic*}.]
\item Diberikan daerah tertutup yang dibatasi oleh parabola $x=2y-2y^{2}$ 
dan sumbu-$y$.
    \begin{enumerate}[label={\alph*}.]
    \item Sketsalah daerah tertutup tersebut!
    \item Daerah tersebut diputar mengelilingi sumbu-$x$. Hitunglah volume 
    benda putar yang dihasilkan dengan menggunakan metode kulit tabung!
    \end{enumerate}
\item 
    \begin{enumerate}[label={\alph*}.]
    \item Tentukan nilai $x$ yang memenuhi 
    $\int_{1/3}^{x}\frac{1}{t}\,dt=2\int_{1}^{x}\frac{1}{t}\,dt$.
    \item Misalkan $\ds f(x)=\frac{a^{x}-1}{a^{x}+1}$, untuk $a$ tertentu dengan $a > 0$, $a\neq 0$.
    Tentukan $f^{-1}$.
    \end{enumerate}
\end{enumerate}


\vspace{0.2cm}\hrule height 1pt\vspace{0.5cm}


\begin{center}
\textbf{\large{PEMBAHASAN}}
\end{center}
\begin{enumerate}[leftmargin=*, label={\arabic*}.]
\item Diberikan daerah tertutup yang dibatasi oleh parabola $x=2y-2y^{2}$ 
dan sumbu-$y$.
    \begin{enumerate}[label={\alph*}.]
        \item Akan disketsa daerah tertutup tersebut.
        
        Gunakan tiga titik dari $x=2y-2y^{2}$ untuk mensketsa kurva.
    \begin{center}
        \begin{tabular}{|c|c|c|c|}\hline
            $y$ & $0$ & $1$ & $2$ \\ \hline
            $x=2y-2y^{2}$ & $0$ & $0$ & $-4$ \\ \hline
        \end{tabular}
    \end{center}
    Diperoleh daerah yang dibatasi seperti berikut:

    \begin{center}
\begin{tikzpicture}
    \realline;

    \draw (0,.1)--(0,-.1) node[below=0.2em]{$1$};
    \node at (-6,.4) {$\abs{x-1}=$};        
    \node at (-2.5,.4) {$-(x-1)$};
    \node at (2.5,.4) {$x-1$};
\end{tikzpicture}
\end{center}

    $\therefore$ Telah disketsa daerah tertutup tersebut.


\begin{center}\line(1,0){150}\end{center}


    \item Akan dicari volume benda putar yang dihasilkan terhadap sumbu-$x$ 
    dengan menggunakan metode kulit tabung.

    Dengan kulit tabung, maka partisi akan sejajar dengan sumbu-$x$. Jari-jari 
    kulit tabung dibatasi diatas dan dibawah oleh titik potong $x=2y-2y^{2}$ dengan 
    sumbu-$y$
    \begin{align*}
        x=0, x=2y-y^{2} &\Longrightarrow 0 = 2y-2y^{2}\\
        &\iff 2y(1-y)=0
    \end{align*}
    Sehingga jari-jari kulit tabung adalah $y$ dimana $0 \leq y \leq 1$. Tinggi 
    kulit tabung adalah jarak dari sumbu-$y$ ke kurva $x=2y-2y^{2}$.

    Gunakan integral untuk memperoleh volume benda putar dengan metode kulit tabung
    \begin{align*}
        V &= \int \text{Luas Kulit Tabung}\\
        &=\int_{0}^{1} 2\pi(\text{jari-jari})(\text{tinggi})\,dy\\
        &=2\pi\int_{0}^{1} (y)\brk*{2y-2y^{2}}\,dx\\
        &=2\pi\int_{0}^{1} 2y^{2}-2y^{3}\,dx\\
        &=2\pi \eval{\frac{2}{3}y^{3}-\frac{1}{2}y^{4}}{0}{1}\\
        &=2\pi \brk*{\brk*{\frac{2}{3}(1)^{3}-\frac{1}{2}(1)^{4}}-(0)}\\
        &=2\pi\frac{1}{6} = \frac{\pi}{3}
    \end{align*}
    Berikut ilustrasi hasil benda padat

    \begin{center}
\begin{tikzpicture}
\begin{axis}[view={-30}{30},colormap/Spectral-9]
    \begin{scope}
        \def\ry{(-1)}
        \def\fx{(2-x)}
        \addplot3[surf,shader=flat,name path=f,
            samples=25,
            domain=0:1,y domain=0:2*pi,
            z buffer=sort]
            (x,{((\fx-\ry) * cos(deg(y))+\ry)}, {((\fx-\ry) * sin(deg(y))+\ry)});
    \end{scope}

    \begin{scope}
        \def\ry{(-1)}
        \def\fx{(x^2)}
        \addplot3[surf,shader=flat,name path=g,
            samples=25,
            domain=0:1,y domain=0:2*pi,
            opacity=0.25,
            z buffer=sort]
            (x,{((\fx-\ry) * cos(deg(y))+\ry)}, {((\fx-\ry) * sin(deg(y))+\ry)});
    \end{scope}

    \addplot3[surf,shader=flat,name path=g,
        samples=25,
        domain=0:2,y domain=0:2*pi,
        z buffer=sort]
        (0,{(x+1)*cos(deg(y))-1},{(x+1)*sin(deg(y))-1});

    \addplot3[black,shader=flat,
            samples=25,
            domain=0:1,
            z buffer=sort]
            (x,-1,-1);
    \node at (0,-1,-1) {$y=-1$};
\end{axis}
\end{tikzpicture}
\end{center}

    $\therefore$ Diperoleh volume benda putar yang dihasilkan terhadap sumbu-$x$ 
    dengan menggunakan metode kulit tabung adalah $\pi/3$.

    \end{enumerate}

\begin{center}\line(1,0){300}\end{center}


\item 
    \begin{enumerate}[label={\alph*}.]
    \item Akan dicari nilai $x$ yang memenuhi 
    $\ds\int_{1/3}^{x}\frac{1}{t}\,dt=2\int_{1}^{x}\frac{1}{t}\,dt$.

    Selesaikan kedua integral.
    \begin{align*}
        &\int_{1/3}^{x}\frac{1}{t}\,dt=2\int_{1}^{x}\frac{1}{t}\,dt\\
        \iff &\eval{\ln\abs{t}}{1/3}{x}=2\eval{\ln\abs{t}}{1}{x}\\
        \iff &\ln\abs{x}-\ln\abs*{\frac{1}{3}}=2\brk*{\ln\abs{x}-\ln\abs{1}}\\
        \iff &\ln\abs{x}-\ln\brk*{\frac{1}{3}}=2\brk*{\ln\abs{x}-0}\\
        \iff &\ln\abs{x}-\ln\brk*{\frac{1}{3}}=2\ln\abs{x}\\
        \iff &-\ln\brk*{\frac{1}{3}}=\ln\abs{x}\\
        \iff &\ln\brk*{\frac{1}{3}}^{-1}=\ln\abs{x}\\
        \iff &\ln(3)=\ln\abs{x}\\
        \iff &3=\abs{x}\\
    \end{align*}
    Dengan demikian, nilai $x$ yang memenuhi adalah $x=3$ atau $x=-3$.

    $\therefore$ Nilai $x$ yang memenuhi 
    $\ds\int_{1/3}^{x}\frac{1}{t}\,dt=2\int_{1}^{x}\frac{1}{t}\,dt$ adalah 
    $x=3$ atau $x=-3$.


\begin{center}\line(1,0){150}\end{center}


    \item Diberikan $\ds f(x)=\frac{a^{x}-1}{a^{x}+1}$, untuk $a$ tertentu dengan $a > 0$, $a\neq 0$.
    Akan dicari $f^{-1}$.

    Misalkan $f(x)=y$, maka
    \begin{align*}
        &y = \frac{a^{x}-1}{a^{x}+1}\\
        \iff & y(a^{x}+1) = a^{x}-1
        &\text{karena }a^{x}+1\neq 0\\
        \iff & ya^{x}+y = a^{x}-1\\
        \iff & y+1 = a^{x}-ya^{x}\\
        \iff & y+1 = a^{x}(1-y)\\
        \iff & a^{x} = \frac{1+y}{1-y}\\
        \iff & x = \log_{a}\brk*{\frac{1+y}{1-y}}\\
    \end{align*}
    Sehingga 
    \[
    f^{-1}(y) = x = \log_{a}\brk*{\frac{1+y}{1-y}}
    \]

    $\therefore$ Diperoleh $\ds f^{-1}(x)=\log_{a}\brk*{\frac{1+x}{1-x}}$

    \end{enumerate}
\end{enumerate}

\begin{center}\line(1,0){300}\end{center}
