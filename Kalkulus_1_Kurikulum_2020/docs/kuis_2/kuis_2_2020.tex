\begin{flushright}
    \section*{\Large{Kuis 2}}
    \addcontentsline{toc}{section}{Kuis 2}
    \subsection*{Tahun 2020}
    \addcontentsline{toc}{subsection}{Kuis 2 - 2020}
\end{flushright}


\vspace{0.5cm}\hrule height 2pt\vspace{0.5cm}


\begin{center}
\textbf{\large{MATERI}}
\begin{enumerate}[leftmargin=*, label={\arabic*}.]
    \item Sketsa grafik fungsi dan kurva.
    \item Mencari nilai integral.
    \item Mencari nilai integral dengan metode subtitusi.
    \item Aplikasi integral dalam mencari luas daerah.
    \item Aplikasi integral dalam mencari volume benda putar.
\end{enumerate}
\end{center}


\vspace{0.2cm}\hrule height 1pt\vspace{0.5cm}


\begin{center}
\textbf{\large{SOAL}}
\end{center}
\begin{enumerate}[leftmargin=*, label={\arabic*}.]
\item Misalkan
\[
f(x) = 
\begin{cases}
    \cos x, & x < 0\\
    1-x^{2}, &x \geq 0
\end{cases}
\]
    \begin{enumerate}[label={\alph*}.]
    \item Gambarlah grafik fungsi $f$ tersebut
    \item Hitunglah $\int_{-\pi/2}^{1}f(x)\,dx$.
    \end{enumerate}
\item Hitunglah integral berikut ini:
    \begin{enumerate}[label={\alph*}.]
    \item $\int_{0}^{\pi/2} \sin x \sin(\cos x) \,dx$
    \item $\int_{2}^{10} x^{4}\cos\brk*{2x^5}\,dx$
    \end{enumerate}
\item Sketsalah daerah yang dibatasi oleh $x+y^2=0$ dan $x+3y^{2}=2$, 
kemudian tentukanlah luas daerahnya.
\item Buatlah sketsa daerah yang dibatasi oleh $y=x+2$ dan $y=x^{2}$, 
kemudian tentukanlah volume benda putar yang terbentuk jika daerah 
tersebut diputar mengelilingi garis $y=4$.
\end{enumerate}


\vspace{0.2cm}\hrule height 1pt\vspace{0.5cm}


\begin{center}
\textbf{\large{PEMBAHASAN}}
\end{center}
\begin{enumerate}[leftmargin=*, label={\arabic*}.]
\item 
    \begin{enumerate}[label={\alph*}.]
    \item Uji tiga titik saat $x\geq 0$ ke fungsi $1-x^{2}$ dan sambungkan dengan 
    fungsi $\cos x$ di $x < 0$.
    \begin{center}
        \begin{tabular}{|c|c|c|c|}\hline
            $x$ & $0$ & $1$ & $2$ \\ \hline
            $f(x)$ & $1$ & $0$ & $-3$ \\ \hline
        \end{tabular}
    \end{center}
    Diperoleh grafik $f$ sebagai berikut.

    \begin{center}
\begin{tikzpicture}[>=stealth]
\begin{axis}[
    xmin=-5,xmax=3,
    ymin=-4,ymax=2,
    axis x line=middle,
    axis y line=middle,
    axis line style=<->,
    xlabel={$x$},
    ylabel={$f(x)$},
    ]
    
    \addplot[no marks,blue, <-] 
        expression[domain=-5:-0.01,samples=50]{cos(deg(x))};
    \addplot[no marks,blue, ->] 
        expression[domain=0:2.2,samples=50]{1-(x^2)};

    \point{black}{(0,1)};
    \point{black}{(1,0)};
    \point{black}{(2,-3)};
\end{axis}
\end{tikzpicture}
\end{center}
    
    $\therefore$ Telah digambar grafik fungsi $f$


\begin{center}\line(1,0){150}\end{center}


    \item Akan dicari $\ds \int_{-\pi/2}^{1}f(x)\,dx$

    \begin{align*}
        \int_{-\pi/2}^{1}f(x)\,dx &=\int_{-\pi/2}^{0} f(x)\,dx + \int_{0}^{1} f(x)\,dx\\
        &=\int_{-\pi/2}^{0} \cos x\,dx + \int_{0}^{1} 1-x^{2}\,dx\\
        &=\eval{\sin x}{-\pi/2}{0} + \eval{x-\frac{1}{3}x^3}{0}{1}\\
        &=\brk*{\sin 0 - \sin\brk*{-\frac{\pi}{2}}}
        +\brk*{\brk*{1-\frac{1}{3}(1)^3}-\brk*{0-\frac{1}{3}(0)^3}}\\
        &=(0-(-1))+\brk*{\frac{2}{3}-0} = \frac{5}{3}
    \end{align*}
    
    $\therefore$ $\ds \int_{-\pi/2}^{1}f(x)\,dx =\frac{5}{3}$

    \end{enumerate}

\begin{center}\line(1,0){300}\end{center}


\item Gunakan metode subtitusi untuk menyelesaikan kedua integral
    \begin{enumerate}[label={\alph*}.]
    \item Akan dicari $\ds \int_{0}^{\pi/2} \sin x\sin(\cos x)\,dx$.
    
    Subtitusi
    \begin{align*}
        &u = \cos x,&\frac{du}{dx} = -\sin x \iff -du = \sin x\,dx\\
        &x=0 \iff u = \cos 0 = 1,&x=\frac{\pi}{2} \iff u = \cos \frac{\pi}{2} = 0
    \end{align*}
    Sehingga 
    \begin{align*}
        &\int_{0}^{\pi/2} \sin x\sin(\cos x)\,dx\\
        =\,& \int_{0}^{\pi/2} \sin(\cos x)\sin x\,dx\\
        =\,& \int_{1}^{0} \sin(u)\,(-du)\\
        =\,& \int_{0}^{1} \sin(u)\,du\\
        =\,& \eval{-\cos u}{0}{1}\\
        =\,& \brk*{-\cos(1)-(-\cos 0)} = 1-\cos 1
    \end{align*}
    
    $\therefore$ $\ds \int_{0}^{\pi/2} \sin x\sin(\cos x)\,dx = 1-\cos 1$


\begin{center}\line(1,0){150}\end{center}


    \item Akan dicari $\ds \int_{2}^{10} x^{4}\cos\brk*{2x^{5}}\,dx$.
    
    Subtitusi
    \begin{align*}
        &u = 2x^{5},&\frac{du}{dx} = 10x^{4} \iff \frac{du}{10} = x^{4}\,dx\\
        &x=2 \iff u = 2\cdot2^{5},&x=10 \iff u = 2\cdot10^{5}
    \end{align*}
    Sehingga 
    \begin{align*}
        &\int_{2}^{10} x^{4}\cos\brk*{2x^{5}}\,dx\\
        =\,& \int_{2}^{10} \cos\brk*{2x^{5}}x^{4}\,dx\\
        =\,& \int_{2\cdot2^{5}}^{2\cdot10^{5}} \cos(u)\,\brk*{\frac{du}{10}}\\
        =\,& \frac{1}{10}\int_{2\cdot2^{5}}^{2\cdot10^{5}} \cos(u)\,du\\
        =\,& \frac{1}{10}\eval{\sin u}{2\cdot2^{5}}{2\cdot10^{5}}\\
        =\,& \frac{\sin \brk*{2\cdot10^{5}}-\sin \brk*{2\cdot2^{5}}}{10}
    \end{align*}

    $\therefore$ $\ds \int_{2}^{10} x^{4}\cos\brk*{2x^{5}}\,dx
    =\frac{\sin \brk*{2\cdot10^{5}}-\sin \brk*{2\cdot2^{5}}}{10}$

    \end{enumerate}

\begin{center}\line(1,0){300}\end{center}


\item Gunakan tiga titik dari masing-masing persamaan untuk mensketsa kurva.
\begin{center}
    \begin{tabular}{|c|c|c|c|}\hline
        $y$ & $-1$ & $0$ & $1$ \\ \hline
        $x=-y^2$ & $-1$ & $0$ & $-1$ \\ \hline
    \end{tabular}\quad
    \begin{tabular}{|c|c|c|c|}\hline
        $y$ & $-1$ & $0$ & $1$ \\ \hline
        $x=2-3y^{2}$ & $-1$ & $2$ & $-1$ \\ \hline
    \end{tabular}
\end{center}
Diperoleh daerah yang dibatasi seperti berikut:

\begin{center}
\begin{tikzpicture}[>=stealth]
\begin{axis}[
    xmin=-3,xmax=3,
    ymin=-2,ymax=2,
    axis x line=middle,
    axis y line=middle,
    axis line style=<->,
    xlabel={$x$},
    ylabel={$y$},
    ]
        
    \addplot [name path=f, no marks,blue, domain=-2:2,samples=50]({-x^2},{x});
    \node [blue] at (-1,0.5){\scalebox{0.7}{$x+y^{2}=0$}};
    \addplot [name path=g, no marks,red, domain=-2:2,samples=50]({2-3*x^2},{x});
    \node [red] at (1,1){\scalebox{0.7}{$x+3y^{2}=2$}};

    \point{blue}{(0,0)};
    \point{red}{(2,0)};
    \point{purple}{(-1,1)};
    \point{purple}{(-1,-1)};
        
    \addplot [thick, color=cyan, fill=cyan, fill opacity=0.5]
        fill between[of=f and g, soft clip={domain=-1:2},]; 
\end{axis}
\end{tikzpicture}
\end{center}

Untuk menghitung luasnya, kedua persamaan akan digunakan dan diintegralkan terhadap $y$, sehingga 
diperlukan batas atas dan bawah $y$. Pada sketsa gambar telah terlihat batas bawahnya adalah $y=-1$ dan 
batas atasnya adalah $y=1$. Bila titik uji berbeda, carilah titik perpotongan kedua persamaan seperti berikut.
\begin{align*}
    x = -y^{2},\quad x = 2-3y^{2} &\Longrightarrow -y^{2}= 2-3y^{2}\\
    &\iff 0 = 2-2y^{2}\\
    &\iff y^{2} = 1
\end{align*}
Sehingga $y = 1$ dan $y=-1$ adalah titik potong kedua kurva.

Gunakan integral untuk menghitung luas
\begin{align*}
    L &= \int_{-1}^{1} \brk*{2-3y^{2}}-\brk*{-y^{2}}\,dy\\
    &=\int_{-1}^{1} 2-2y^{2}\,dy\\
    &=\eval{2y-\frac{2}{3}y^{3}}{-1}{1}\\
    &=\brk*{2(1)-\frac{2}{3}(1)^{3}}-\brk*{2(-1)-\frac{2}{3}(-1)^{3}}\\
    &=\frac{8}{3}
\end{align*}
$\therefore$ Luas daerah yang dibatasi kedua kurva adalah $\ds\frac{8}{3}$.


\begin{center}\line(1,0){300}\end{center}


\item Uji dua titik untuk sketsa $y=x+2$ dan tiga titik untuk $y=x^2$.
\begin{center}
    \begin{tabular}{|c|c|c|}\hline
        $x$  & $0$ & $1$ \\ \hline
        $y=x+2$ & $2$ & $3$ \\ \hline
    \end{tabular}\quad
    \begin{tabular}{|c|c|c|c|}\hline
        $x$ & $-1$ & $0$ & $1$ \\ \hline
        $y=x^{2}$ & $1$ & $0$ & $1$ \\ \hline
    \end{tabular}
\end{center}
Diperoleh daerah yang dibatasi seperti berikut.

\begin{center}
\begin{tikzpicture}[>=stealth]
\begin{axis}[
    xmin=-3,xmax=3,
    ymin=-0.5,ymax=5,
    axis x line=middle,
    axis y line=middle,
    axis line style=<->,
    xlabel={$x$},
    ylabel={$y$},
    ]
        
    \addplot [name path=f, no marks,blue, domain=-3:3,samples=50]({x},{x+2});
    \node [blue] at (-2,1){\scalebox{0.7}{$y=x+2$}};
    \addplot [name path=g, no marks,red, domain=-3:3,samples=50]({x},{x^2});
    \node [red] at (1.5,1){\scalebox{0.7}{$y=x^{2}$}};
    \addplot [no marks,black, domain=-3:3,samples=50]({x},{4});
    \node [black] at (1,4.3){\scalebox{0.7}{$y=4$}};

    \point{blue}{(0,2)};
    \point{blue}{(1,3)};
    \point{red}{(-1,1)};
    \point{red}{(0,0)};
    \point{red}{(1,1)};
        
    \addplot [thick, color=cyan, fill=cyan, fill opacity=0.5]
        fill between[of=f and g, soft clip={domain=-1:2},]; 
\end{axis}
\end{tikzpicture}
\end{center}

Selanjutnya akan dicari volume benda putar yang terbentuk jika daerah tersebut diputar 
mengelilingi garis $y=4$.\\
Gunakan metode cincin. Dari gambar sebelumnya, bila diputar mengelilingi garis $y=4$, maka bagian 
luarnya adalah jarak dari $y=4$ ke $y=x^{2}$ yaitu $4-x^{2}$ dan bagian dalamnya adalah jarak 
dari $y=4$ ke $y=x+2$ yaitu jari-jari $4-(x+2)=2-x$. 

Metode cincin mempartisi daerah volume secara tegak lurus terhadap $y=4$ 
sehingga integrasi dilakukan terhadap $dx$. Sebelum melakukan integrasi, akan dicari titik potong 
kedua persamaan.
\begin{align*}
    y = x+2,\quad y = x^{2} &\Longrightarrow x+2= x^{2}\\
    &\iff 0 = x^{2}-x-2\\
    &\iff 0 = (x-2)(x+1)
\end{align*}
Sehingga $x = -1$ dan $y=2$ adalah titik potong kedua persamaan juga menjadi batas integrasi.

Gunakan integral untuk memperoleh volume
\begin{align*}
    V &= \int \text{Luas bagian luar} - \text{Luas bagian dalam}\\
    &=\int_{-1}^{2} \pi(4-x^2)^2 - \pi(2-x)^{2}\,dx \\
    &=\pi\int_{-1}^{2} \brk*{x^{4}-8x^{2}+16} - \brk*{x^{2}-4x+4}\,dx \\
    &=\pi\int_{-1}^{2} \brk*{x^{4}-9x^{2}+4x+12}\,dx \\
    &=\pi\eval{\frac{1}{5}x^{5}-3x^{3}+2x^{2}+12x}{-1}{2} \\
    &= \brk*{\frac{72}{5}-\brk*{-\frac{36}{5}}}=\frac{108}{5}\pi
\end{align*}
Berikut ilustrasi hasil benda padat

\begin{center}
\begin{tikzpicture}
\begin{axis}[view={-30}{30},colormap/Spectral-9]
    \begin{scope}
        \def\ry{(4)}
        \def\fx{(x^2)}
        \addplot3[surf,shader=flat,
            samples=25,
            domain=-1:2,y domain=0:2*pi,
            z buffer=sort]
            (x,{((\fx-\ry) * cos(deg(y))+\ry)}, {((\fx-\ry) * sin(deg(y))+\ry)});
    \end{scope}
    
    \begin{scope}
        \def\ry{(4)}
        \def\fx{(x+2)}
        \addplot3[surf,shader=flat,
            samples=25,
            domain=-1:2,y domain=0:2*pi,
            opacity=0.25,
            z buffer=sort]
            (x,{((\fx-\ry) * cos(deg(y))+\ry)}, {((\fx-\ry) * sin(deg(y))+\ry)});
    \end{scope}

    \addplot3[black,shader=flat,
        samples=25,
        domain=-1:2,
        z buffer=sort]
        (x,4,4);
    \node at (-1,5,4) {$y=4$};
\end{axis}
\end{tikzpicture}       
\end{center}

$\therefore$ Telah disketsa daerah yang dibatasi oleh $y=x+2$ dan $y=x^{2}$ lalu volume 
benda putar yang terbentuk jika daerah tersebut diputar mengelilingi garis $y=4$ adalah 
$\frac{108}{5}\pi$ satuan.

\end{enumerate}

\begin{center}\line(1,0){300}\end{center}
