\begin{flushright}
    \section*{\Large{Kuis 1}}
    \addcontentsline{toc}{section}{Kuis 1}
    \subsection*{Tahun 2020}
    \addcontentsline{toc}{subsection}{Kuis 1 - 2020}
\end{flushright}


\vspace{0.5cm}\hrule height 2pt\vspace{0.5cm}


\begin{center}
\textbf{\large{MATERI}}
\begin{enumerate}[leftmargin=*, label={\arabic*}.]
\item Menyelesaikan pertidaksamaan yang melibatkan nilai mutlak.
\item Mencari nilai limit kiri dan nilai limit kanan fungsi.
\item Mencari nilai limit fungsi di tak hingga.
\item Mencari nilai turunan fungsi menggunakan definisi.
\item Menyelesaikan masalah turunan implisit.
\item Menyelesaikan permasalahan laju yang berkaitan.
\item Mencari nilai maksimum dan minimum fungsi.
\end{enumerate}
\end{center}


\vspace{0.2cm}\hrule height 1pt\vspace{0.5cm}


\begin{center}
\textbf{\large{SOAL}}
\end{center}
\begin{enumerate}[leftmargin=*, label={\arabic*}.]
\item Tentukanlah himpunan penyelesaian dari $\abs{2x-1} \geq 2\abs{x+1}$.
\item Hitunglah limit berikut (jika ada)!
    \begin{enumerate}[label={\alph*}.]
    \item $\ds \lim_{x\to -\infty} \frac{2x}{\abs{x}}$
    \item $\ds \lim_{x\to 5^{-}} \frac{x^{2}-3x-10}{x^{2}-10x+25}$
    \end{enumerate}
\item Diberikan fungsi 
$\ds f(x) = 
\begin{cases}
    x+1, &\text{jika $x \geq 0$}\\
    x^{2}+1, &\text{jika $x < 0$}
\end{cases}$. 
Hitunglah $f'(0)$ (jika ada)!
\item Tentukanlah kemiringan garis singgung dari kurva $\cos (xy^{2})-\sin y = x$ 
di titik $(x,y) = (1,0)$.
\item Pada suatu empang yang tenang (tidak ada riak gelombang), dijatuhkan sebuah 
batu kecil, sehingga dihasilkan riak gelombang berbentuk lingkaran. Jari-jari 
riak-lingkaran tersebut bertambah dengan laju konstan 3 cm/detik. Seberapa cepat 
pertambahan \textbf{luas daerah} yang dijalani riak-lingkaran tersebut ketika 
jari-jari riak-lingkaran tersebut 10 cm?
\item Tentukanlah nilai maksimum dan nilai minimum fungsi $f(t)=t\sqrt{4-t^{2}}$ 
pada interval $I = \cic*{-1,2}$
\end{enumerate}


\vspace{0.2cm}\hrule height 1pt


\newpage
\begin{center}
\textbf{\large{PEMBAHASAN}}
\end{center}
\begin{enumerate}[leftmargin=*, label={\arabic*}.]
\item Akan dicari himpunan penyelesaian dari $\abs{2x-1} \geq 2\abs{x+1}$.

Hanya ada 3 aturan dasar yang diperlukan untuk menyelesaikan pertidaksamaan.
    \begin{enumerate}[label={\arabic*})]
    \item Jumlahkan kedua ruas dengan bilangan yang sama.
    \item Kalikan kedua ruas dengan bilangan positif yang sama.
    \item Kalikan kedua ruas dengan bilangan negatif yang sama dan 
    mengubah arah pertidaksamaannya.
    \end{enumerate}
Pada soal ini kedua ruas melibatkan nilai mutlak. Untuk menyelesaikannya 
pertidaksamaan perlu diubah kebentuk yang tidak melibatkan nilai mutlak. 
Cara yang selalu bisa digunakan adalah dengan membagi kasus pada nilai $x$.

Perhatikan bahwa sesuai definisi nilai mutlak
\[
\abs{2x-1} = 
\begin{cases}
    2x-1, &\text{jika $2x-1 \geq 0$ atau $x \geq \frac{1}{2}$}\\
    -(2x-1), &\text{jika $2x-1 < 0$ atau $x < \frac{1}{2}$}
\end{cases}
\]
dan
\[
\abs{x+1} = 
\begin{cases}
    x+1, &\text{jika $x+1 \geq 0$ atau $x \geq -1$}\\
    -(x+1), &\text{jika $x+1 < 0$ atau $x < -1$}
\end{cases}
\]
Sehingga pertidaksamaan ini diselesaikan dengan membagi menjadi 3 kasus seperti 
yang terlihat pada garis bilangan di bawah ini.

\begin{center}
\begin{tikzpicture}[>=stealth]
\begin{axis}[
    xmin=-5,xmax=3,
    ymin=-4,ymax=2,
    axis x line=middle,
    axis y line=middle,
    axis line style=<->,
    xlabel={$x$},
    ylabel={$f(x)$},
    ]
    
    \addplot[no marks,blue, <-] 
        expression[domain=-5:-0.01,samples=50]{cos(deg(x))};
    \addplot[no marks,blue, ->] 
        expression[domain=0:2.2,samples=50]{1-(x^2)};

    \point{black}{(0,1)};
    \point{black}{(1,0)};
    \point{black}{(2,-3)};
\end{axis}
\end{tikzpicture}
\end{center}

\textbf{Kasus 1: $x < -1$}\\
Maka
\begin{align*}
    \abs{2x-1} \geq 2\abs{x+1} 
    \iff &-(2x-1) \geq 2(-(x+1)) 
    &\text{definisi nilai mutlak} \\
    \iff &1-2x \geq -2x-2
    &\text{penyederhanaan}\\
    \iff &1 \geq -2
    &\text{kedua ruas jumlahkan $2x$}
\end{align*}
Ini menandakan untuk semua nilai $x < -1$ pertidaksamaan akan ujungnya 
berbentuk $1 \geq -2$ yang selalu bernilai benar. Dengan kata lain, semua 
nilai $x < -1$ memenuhi pertidaksamaan.

\textbf{Kasus 2: $-1 \leq x < \frac{1}{2}$}\\
Maka
\begin{align*}
    \abs{2x-1} \geq 2\abs{x+1} \iff &-(2x-1) \geq 2(x+1)
    &\text{definisi nilai mutlak} \\
    \iff &1-2x \geq 2x+2
    &\text{penyederhanaan}\\
    \iff &-4x \geq 1
    &\text{kedua ruas jumlahkan $-1-2x$}\\
    \iff &x \leq -\frac{1}{4}
    &\text{kedua ruas kalikan $-\frac{1}{4}$}
\end{align*}
Sehingga untuk kasus $-1 \leq x < \frac{1}{2}$ nilai $x$ yang memenuhi 
pertidaksamaan adalah $x \leq -\frac{1}{4}$. Dengan kata lain, nilai 
$-1 \leq x \leq -\frac{1}{4}$ memenuhi pertidaksamaan.

\textbf{Kasus 3: $x \geq \frac{1}{2}$}\\
Maka
\begin{align*}
    \abs{2x-1} \geq 2\abs{x+1} 
    \iff &2x-1 \geq 2(x+1)
    &\text{definisi nilai mutlak} \\
    \iff &2x-1 \geq 2x+2
    &\text{penyederhanaan}\\
    \iff &-1 \geq 2
    &\text{kedua ruas jumlahkan $-2x$}
\end{align*}
Ini menandakan untuk semua nilai $x \geq \frac{1}{2}$ pertidaksamaan 
akan ujungnya berbentuk $-1 \geq 2$ yang selalu bernilai salah.
Dengan kata lain, semua nilai $x \geq \frac{1}{2}$ tidak memenuhi 
pertidaksamaan.

Himpunan penyelesaiannya adalah gabungan semua nilai $x$ yang memenuhi 
pertidaksamaan. Sehingga himpunan tersebut adalah 
$\set*{x < -1 \cup -1 \leq x \leq -\frac{1}{4}}$ 
atau $\set*{x \in \mathbb{R} \mid x \leq -\frac{1}{4}}$
atau $\oic*{-\infty,-\frac{1}{4}}$

$\therefore$ Himpunan penyelesaian dari $\abs{2x-1} \geq 2\abs{x+1}$
adalah $\set*{x \in \mathbb{R} \mid x \leq -\frac{1}{4}}$
atau $\oic*{-\infty,-\frac{1}{4}}$

\vspace{0.1cm}
\textbf{Catatan:}\\
Salah satu cara untuk mengubah pertidaksamaan yang melibatkan nilai mutlak ke 
pertidaksamaan yang tidak adalah dengan menguadratkan kedua ruas. Soal ini 
dapat diselesaikan dengan cara tersebut tetapi tidak semua soal dapat 
diselesaikan dengan cara itu. Hal ini dikarenakan cara tersebut memiliki syarat 
yaitu kedua ruasnya harus bernilai positif.


\begin{center}\line(1,0){300}\end{center}


\item Akan dicari limit dari soal yang diberikan (jika ada).
    \begin{enumerate}[label={\alph*}.]
    \item Akan dicari limit dari $\ds \lim_{x\to -\infty} \frac{2x}{\abs{x}}$
    
    \begin{align*}
        \lim_{x\to -\infty} \frac{2x}{\abs{x}} 
        &= \lim_{x\to -\infty} \frac{2x}{-x}
        &\text{definisi nilai mutlak $x < 0$ karena menuju $-\infty$}\\
        &= \lim_{x\to -\infty} \frac{2}{-1}
        &\text{kalikan $\frac{1/x}{1/x}$ (karena $x \neq 0$)}\\
        &= -2
        &\text{teorema utama limit}
    \end{align*}
    Sehingga nilai limit fungsi ini saat $x$ menuju $-\infty$ adalah $-2$

    $\therefore$ $\ds \lim_{x\to -\infty} \frac{2x}{\abs{x}} = -2$

    \begin{center}\line(1,0){150}\end{center}

    \item Akan dibuktikan 
    $\ds \lim_{x\to 5^{-}} \frac{x^{2}-3x-10}{x^{2}-10x+25}$ tidak ada.

    \begin{align*}
        \lim_{x\to 5^{-}} \frac{x^{2}-3x-10}{x^{2}-10x+25} 
        &= \lim_{x\to 5^{-}} \frac{(x-5)(x+2)}{(x-5)(x-5)}
        &\text{faktorisasi}\\
        &= \lim_{x\to 5^{-}} \frac{x+2}{x-5}
        &\text{kalikan $\frac{1/(x-5)}{1/(x-5)}$ (karena $x \neq 5$)}\\
        &= -\infty
        &\text{menuju bentuk $\frac{7}{0}$ dari kiri}
    \end{align*}
    Sehingga limitnya tidak ada.

    $\therefore$
    $\ds \lim_{x\to 5^{-}} \frac{x^{2}-3x-10}{x^{2}-10x+25} = -\infty$ 
    (tidak ada)
\end{enumerate}

\vspace{0.1cm}
\textbf{Catatan}:\\
Teorema-teorema limit yang sering digunakan ada pada Bab 1.3 buku rujukan 
\cite{valberg}


\begin{center}\line(1,0){300}\end{center}


\item Akan dicari $f'(0)$ (jika ada) dengan.
\[
f(x) = 
\begin{cases}
    x+1, &\text{jika $x \geq 0$}\\
    x^{2}+1, &\text{jika $x < 0$}
\end{cases}
\]
Gunakan definisi limit dari turunan dimana
\[
f'(x) = \lim_{h\to 0} \frac{f(x+h)-f(x)}{h}
\]
Karena $f$ adalah fungsi \textit{pointwise} yang belum tentu memiliki turunan 
di $x=0$, akan dicari $f'(0)$ dengan mencari limit kiri dan limit kanan definisi 
limit turunannya.

Akan dicari $\ds \lim_{h\to 0^{-}} \frac{f(0+h)-f(0)}{h}$
\begin{align*}
    \lim_{h\to 0^{-}} \frac{f(0+h)-f(0)}{h}
    &= \lim_{h\to 0^{-}} \frac{f(h)-(0+1)}{h}
    &\text{subtitusi ke dalam fungsi}\\
    &= \lim_{h\to 0^{-}} \frac{h^{2}+1-1}{h}
    &\text{subtitusi ke dalam fungsi}\\
    &= \lim_{h\to 0^{-}} \frac{h^{2}}{h}
    &\text{sederhanakan}\\
    &= \lim_{h\to 0^{-}} h = 0
    &\text{kalikan $\frac{1/h}{1/h}$}
\end{align*}

Selanjutnya dicari $\ds \lim_{h\to 0^{+}} \frac{f(0+h)-f(0)}{h}$
\begin{align*}
    \lim_{h\to 0^{+}} \frac{f(0+h)-f(0)}{h}
    &= \lim_{h\to 0^{+}} \frac{f(h)-(0+1)}{h}
    &\text{subtitusi ke dalam fungsi}\\
    &= \lim_{h\to 0^{+}} \frac{h+1-1}{h}
    &\text{subtitusi ke dalam fungsi}\\
    &= \lim_{h\to 0^{+}} \frac{h}{h}
    &\text{sederhanakan}\\
    &= \lim_{h\to 0^{+}} 1 = 1
    &\text{kalikan $\frac{1/h}{1/h}$}
\end{align*}

Perhatikan bahwa
\[
\lim_{h\to 0^{-}} \frac{f(0+h)-f(0)}{h} = 0 
\neq 1 = \lim_{h\to 0^{+}} \frac{f(0+h)-f(0)}{h}
\]
sehingga $\ds \lim_{h\to 0} \frac{f(0+h)-f(0)}{h} = f'(0)$ tidak ada.

$\therefore$ $f'(0)$ tidak ada.


\begin{center}\line(1,0){300}\end{center}


\item Akan dicari kemiringan garis singgung dari kurva 
$\cos (xy^{2})-\sin y = x$ di titik $(x,y) = (1,0)$.

Kemiringan garis singgung kurva dapat dicari dengan mencari nilai turunan dari kurva 
pada titik singgung. Pada soal ini titik tersebut adalah $(x,y) = (1,0)$, sehingga 
cukup mencari nilai turunan pertama kurva pada titik tersebut.\\
Gunakan metode untuk mencari turunan secara implisit dan aturan rantai.
\begin{align*}
    &\drv{x}{\cos (xy^{2})-\sin y} = \drv{x}{x}\\
    \iff &\drv{x}{\cos (xy^{2})} - \drv{x}{\sin y} = 1
    &\text{sifat linear turunan}\\
    \iff &\brk*{-\sin(xy^{2})\drv{x}{xy^{2}}}
    -\brk*{\cos y \drv{x}{y}} = 1
    &\text{aturan rantai}\\
    \iff &\brk*{-\sin (xy^{2})\brk*{y^{2}+xyy'}} 
    -\brk*{\cos (y) y'} = 1
    &\text{aturan perkalian}
\end{align*}
Sekarang subtitusi $(x,y)=(1,0)$ untuk memperoleh nilai $y'$.
\begin{align*}
    &\brk*{-\sin (xy^{2})\brk*{y^{2}+xyy'}} - \brk*{\cos (y) y'} = 1\\
    \iff &\brk*{-\sin (1\cdot 0^{2})\brk*{0^{2}+1\cdot 0\cdot y'}} - \brk*{\cos (0) y'} = 1\\
    \iff &\brk*{-\sin(0)(0)} - \brk*{(1)y'} = 1\\
    \iff &-y' = 1 \iff y'=-1
\end{align*}
Diperoleh nilai turunan di titik $(x,y) = (1,0)$ adalah $-1$ sehingga kemiringan 
garis singgung kurva di titik tersebut adalah $-1$

$\therefore$ Kemiringan garis singgung kurva pada titik $(x,y) = (1,0)$ adalah $-1$.


\begin{center}\line(1,0){300}\end{center}


\item Akan dicari cepat pertambahan luas daerah yang dijalani riak-lingkaran ketika 
jari-jarinya $10$ cm.

Misalkan $L=$ luas daerah riak-lingkaran dan $r=$ jari-jari riak-lingkaran.\\
Informasi yang diberikan soal adalah cepat pertambahan jari-jari riak-lingkaran yang 
konstan sehingga 
\[\drv{t}{r} = 3\]
Cepat pertambahan luas daerah adalah $\drvL{t}{L}$. Karena $L$ adalah luas 
riak-lingakaran maka 
\begin{center}
    $L = \pi r^{2}$ dan $\ds \drv{r}{L} = 2\pi r$
\end{center}
Dengan aturan rantai maka
\[
\drv{t}{L} = \drv{r}{L}\cdot\drv{t}{r} = 2\pi r \cdot 3 = 6\pi r
\]
Sehingga $\drvL{t}{L}$ saat $r=10$ adalah $60\pi$

$\therefore$ Cepat pertambahan luas daerah yang dijalani riak-lingkaran ketika 
jari-jarinya $10$ cm adalah \\
$60\pi$ cm$^{2}$/detik.


\begin{center}\line(1,0){300}\end{center}


\item 
Akan dicari nilai maksimum dan nilai minimum fungsi $f(t)=t\sqrt{4-t^{2}}$ pada 
interval $\cic*{-1,2}$.

Carilah titik kritis dan uji nilai fungsi pada setiap titik tersebut untuk menemukan 
titik ekstrim.\\
Titik kritis terjadi pada
    \begin{enumerate}[label={\arabic*})]
    \item Ujung interval
    \item Titik stasioner (saat $f'(c)=0$)
    \item Titik singular (saat $f'(c)$ tidak ada)
    \end{enumerate}
Dengan demikian ujung interval $x=-1,2$ adalah titik kritis. Karena $f'(c)$ ada 
untuk semua $x$ pada interval $\cic*{-1,2}$ maka tidak ada titik singular. Terakhir akan 
dicari titik stasioner.\\
Pertama akan dicari $f'(t)$ 
\begin{align*}
    f'(t) = \drv{t}{t\sqrt{4-t^{2}}} 
    &= \drv{t}{t\brk*{4-t^{2}}^{1/2}}
    &\text{penyederhanaan}\\
    &= \brk*{4-t^2}^{1/2}+t\brk*{\frac{1}{2}\brk*{4-t^{2}}^{-1/2}\cdot\drv{t}{4-t^{2}}}
    &\text{aturan perkalian \& rantai}\\
    &= \sqrt{4-t^{2}}+t\brk*{\frac{1}{2\sqrt{4-t^{2}}}\cdot (-2t)}
    &\text{penyederhanaan}\\
    &= \frac{4-t^{2}-t^{2}}{\sqrt{4-t^{2}}}
    &\text{penyederhanaan}\\
    &= \frac{4-2t^{2}}{\sqrt{4-t^{2}}}
    &\text{penyederhanaan}
\end{align*}
Sehingga saat $f'(c)=0$ maka
\begin{align*}
    f'(c) &= \frac{4-2c^{2}}{\sqrt{4-c^{2}}} = 0\\
    \iff &4-2c^{2} = 0
    &\text{kedua ruas kalikan $\sqrt{4-c^{2}}$}\\
    \iff &-2c^{2} = -4
    &\text{kedua ruas ditambah $-4$}\\
    \iff &c^2 = 2
    &\text{kedua ruas kalikan $-\frac{1}{2}$}\\
\end{align*}
Nilai $c$ yang memenuhi adalah $c=-\sqrt{2}, \sqrt{2}$. Abaikan $-\sqrt{2}$ karena 
tidak terletak di interval $\cic{-1,2}$ dan diperoleh titik stasioner 
$c = \sqrt{2}$.\\
Diperoleh tiga titik kritis $x = -1,\sqrt{2},2$.\\
Evaluasi fungsi pada ketiga titik tersebut diperoleh $f(-1) = -\sqrt{3}$, 
$f(\sqrt{2})=2$ dan $f(2) = 0$. Dengan demikian, titik maksimum terjadi pada 
$x = \sqrt{2}$ dan titik minimum terjadi pada $x = -1$

$\therefore$ Nilai maksimum dan nilai minimum fungsi $f(t)=t\sqrt{4-t^2}$ pada 
interval $\cic{-1,2}$ \\
adalah $2$ $\brk*{\text{saat $x = \sqrt{2}$}}$ 
dan $-\sqrt{3}$ $\brk*{\text{saat $x = -1$}}$.

\end{enumerate}

\begin{center}\line(1,0){300}\end{center}
