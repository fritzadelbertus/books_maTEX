\begin{flushright}
    \textbf{\Large{Kuis 1}}
    \subsection*{Tahun 2023}
    \addcontentsline{toc}{subsection}{Kuis 1 - 2023}
\end{flushright}


\vspace{0.5cm}\hrule height 2pt\vspace{0.5cm}


\begin{center}
\textbf{\large{MATERI}}
\begin{enumerate}[leftmargin=*, label={\arabic*}.]
\item Menyelesaikan pertidaksamaan yang melibatkan nilai mutlak.
\item Fungsi komposisi dan menentukan domainnya.
\item Mencari nilai limit kiri dan nilai limit kanan fungsi.
\item Menentukan kekontinuan fungsi pada suatu titik.
\end{enumerate}
\end{center}


\vspace{0.2cm}\hrule height 1pt\vspace{0.5cm}


\begin{center}
\textbf{\large{SOAL}}
\end{center}
\begin{enumerate}[leftmargin=*, label={\arabic*}.]
\item Misalkan $f(x) = x^{2}-x$ dan $g(x)=\abs{x-1}$.
    \begin{enumerate}[label={\alph*}.]
    \item Tentukanlah nilai $x$ yang memenuhi $f(x) > g(x)$.
    \item Jika $\ds h(x) = \frac{g(x)}{f(x)}$, maka tentukanlah 
    domain dari $h(x)$. Jelaskanlah jawaban anda!
    \end{enumerate}
\item Misalkan
\[
    f(x) =
    \begin{cases}
        c^{2}x^{2}, &{x < 2},\\
        (1-c)x, &{x \geq 2}.
    \end{cases}
\]
Tentukan nilai $c$ agar $f$ kontinu di setiap bilangan real $x$.
\end{enumerate}


\vspace{0.2cm}\hrule height 1pt\vspace{0.5cm}


\begin{center}
\textbf{\large{PEMBAHASAN}}
\end{center}
\begin{enumerate}[leftmargin=*, label={\arabic*}.]
\item
    \begin{enumerate}[label={\alph*}.]
    \item Akan dicari nilai $x$ yang memenuhi
    $f(x) > g(x) \iff x^{2}-x > \abs{x-1}$.

    \vspace{0.1cm}
    Hanya ada 3 aturan dasar yang diperlukan untuk menyelesaikan pertidaksamaan.
        \begin{enumerate}[label={\arabic*})]
        \item Jumlahkan kedua ruas dengan bilangan yang sama.
        \item Kalikan kedua ruas dengan bilangan positif yang sama.
        \item Kalikan kedua ruas dengan bilangan negatif yang sama dan 
        mengubah arah pertidaksamaannya.
        \end{enumerate}
    Pada soal ini ruas kanan melibatkan nilai mutlak. Untuk menyelesaikannya 
    pertidaksamaan perlu diubah kebentuk yang tidak melibatkan nilai mutlak. Cara yang 
    selalu bisa digunakan adalah dengan membagi kasus pada nilai $x$.
        
    Perhatikan bahwa sesuai definisi nilai mutlak
    \[
    \abs{x-1} = 
    \begin{cases}
        x-1, &\text{jika $x-1 \geq 0$ atau $x \geq 1$}\\
        -(x-1), &\text{jika $x-1 < 0$ atau $x < 1$}
    \end{cases}
    \]
    Sehingga pertidaksamaan ini diselesaikan dengan membagi menjadi 2 kasus 
    seperti yang terlihat pada garis bilangan di bawah ini.

    \begin{center}
\begin{tikzpicture}
    \realline;

    \draw (0,.1)--(0,-.1) node[below=0.2em]{$1$};
    \node at (-6,.4) {$\abs{x-1}=$};        
    \node at (-2.5,.4) {$-(x-1)$};
    \node at (2.5,.4) {$x-1$};
\end{tikzpicture}
\end{center}
        
    \textbf{Kasus 1: $x < 1$}\\
    Maka
    \begin{align*}
        x^{2}-x > \abs{x-1}
        \iff &x^{2}-x > -(x-1)
        &\text{definisi nilai mutlak} \\
        \iff &x^{2}-x > 1-x
        &\text{penyederhanaan}\\
        \iff &x^{2} - 1 > 0
        &\text{kedua ruas jumlahkan $x-1$}\\
        \iff &(x+1)(x-1) > 0
        &\text{faktorisasi}
    \end{align*}
    Maka $x=1$ dan $x=-1$ adalah titik stasioner.

    \begin{center}
\begin{tikzpicture}
\begin{axis}[view={-30}{30},colormap/Spectral-9]
    \begin{scope}
        \def\ry{(-1)}
        \def\fx{(2-x)}
        \addplot3[surf,shader=flat,name path=f,
            samples=25,
            domain=0:1,y domain=0:2*pi,
            z buffer=sort]
            (x,{((\fx-\ry) * cos(deg(y))+\ry)}, {((\fx-\ry) * sin(deg(y))+\ry)});
    \end{scope}

    \begin{scope}
        \def\ry{(-1)}
        \def\fx{(x^2)}
        \addplot3[surf,shader=flat,name path=g,
            samples=25,
            domain=0:1,y domain=0:2*pi,
            opacity=0.25,
            z buffer=sort]
            (x,{((\fx-\ry) * cos(deg(y))+\ry)}, {((\fx-\ry) * sin(deg(y))+\ry)});
    \end{scope}

    \addplot3[surf,shader=flat,name path=g,
        samples=25,
        domain=0:2,y domain=0:2*pi,
        z buffer=sort]
        (0,{(x+1)*cos(deg(y))-1},{(x+1)*sin(deg(y))-1});

    \addplot3[black,shader=flat,
            samples=25,
            domain=0:1,
            z buffer=sort]
            (x,-1,-1);
    \node at (0,-1,-1) {$y=-1$};
\end{axis}
\end{tikzpicture}
\end{center}

    Nilai yang memenuhi adalah $x < -1$ atau $x > 1$.
    Karena syarat kasus ini adalah $x < 1$ maka nilai $x$ yang memenuhi adalah
    $x < -1$.

    \textbf{Kasus 2: $x \geq 1$}\\
    Maka
    \begin{align*}
        x^{2}-x > \abs{x-1}
        \iff &x^{2}-x > x-1
        &\text{definisi nilai mutlak} \\
        \iff &x^{2} -2x + 1> 0
        &\text{kedua ruas jumlahkan $-x+1$}\\
        \iff &(x-1)(x-1) > 0
        &\text{faktorisasi}
    \end{align*}
    Maka $x=-1$ dan $x=-1$ adalah titik stasioner.

    \begin{center}
\begin{tikzpicture}
\begin{axis}[view={-30}{30},colormap/Spectral-9]
    \addplot3[surf,shader=flat,name path=g,
        samples=25,
        domain=0:2,y domain=0:2*pi,
        z buffer=sort]
        ({3*cos(deg(y))+3},{3*sin(deg(y))+3},x);

    \begin{scope}
        \def\rx{(3)}
        \def\fx{(x^2)}
        \addplot3[surf,shader=flat,name path=g,
            samples=25,
            domain=0:1,y domain=0:2*pi,
            opacity=0.25,
            z buffer=sort]
            ({(((x-\rx) * cos(deg(y)))+\rx)}, {(((x-\rx) * sin(deg(y)))+\rx)}, {\fx});
    \end{scope}

    \begin{scope}
        \def\rx{(3)}
        \def\fx{(2-x)}
        \addplot3[surf,shader=flat,name path=f,
            samples=25,
            domain=0:1,y domain=0:2*pi,
            opacity=0.25,
            z buffer=sort]
            ({(((x-\rx) * cos(deg(y)))+\rx)}, {(((x-\rx) * sin(deg(y)))+\rx)}, {\fx});
    \end{scope}

    \addplot3[black,shader=flat,
            samples=25,
            domain=0:2,
            z buffer=sort]
            (3,3,x);
    \node at (3,1,2) {$x=3$};
\end{axis}
\end{tikzpicture}
\end{center}

    Nilai yang memenuhi adalah $x \neq 1$.
    Karena syarat kasus ini adalah $x \geq 1$ maka nilai $x$ yang memenuhi adalah
    $x > 1$.
        
    Semua nilai $x$ yang memenuhi $f(x) > g(x)$ adalah gabungan nilai $x$ 
    dari kedua kasus. Sehingga nilai $x$ yang memenuhi adalah 
    $\set*{x \in \mathbb{R} \mid x < -1 \cup x > 1}$
    atau $\oic*{-\infty, -1} \cup \cio*{1, \infty}$
        
    $\therefore$ Nilai $x$ yang memenuhi $f(x) > g(x)$ adalah
    $\set*{x \in \mathbb{R} \mid x < -1 \cup x > 1}$
    atau $\oic*{-\infty, -1} \cup \cio*{1, \infty}$


\begin{center}\line(1,0){150}\end{center}


    \item Akan dicari domain dari $\ds h(x) = \frac{g(x)}{f(x)}$.
    
    Dari definisi tersebut ini adalah fungsi komposisi dengan 
    \[
        h(x) = \frac{g(x)}{f(x)} = \frac{\abs{x-1}}{x^{2}-x}
    \]
    Sehingga domainnya adalah irisan dari domain $f$, $g$, dan 
    $\ds \frac{\abs{x-1}}{x^{2}-x}$.\\
    Domain dari $f$ dan $g$ adalah $\mathbb{R}$, sehingga cukup mencari 
    domain dari $\ds \frac{\abs{x-1}}{x^{2}-x}$. Karena melibatkan 
    bentuk rasional maka penyebut tidak boleh nol. Sehingga
    \begin{align*}
        x^{2}-x \neq 0 \iff &x(x-1)\neq 0
        &\text{faktorisasi}
    \end{align*}
    Sehingga $x \neq 1$ dan $x \neq 0$. \\Domain dari 
    $\ds \frac{\abs{x-1}}{x^{2}-x}$ adalah 
    $\set*{x \in \mathbb{R} \mid x \neq 1, x\neq 0}$.\\ Irisan dari ketiga domain
    $\mathbb{R}$, $\mathbb{R}$, dan $\set*{x \in \mathbb{R} \mid x \neq 1, x\neq 0}$ 
    adalah $\set*{x \in \mathbb{R} \mid x \neq 1, x\neq 0}$.\\
    Dengan demikian domain dari $h$ adalah 
    $\set*{x \in \mathbb{R} \mid x \neq 1, x\neq 0}$.

    $\therefore$ Domain dari fungsi $h$ adalah 
    $\set*{x \in \mathbb{R} \mid x \neq 1, x\neq 0}$

    \end{enumerate}

\begin{center}\line(1,0){300}\end{center}


\item
    \begin{enumerate}[label={\alph*}.]
    \item \[
    f(x) = 
    \begin{cases}
        c^{2}x^{2}, &{x < 2},\\
        (1-c)x, &{x \geq 2}.
    \end{cases}
    \]
    Akan dicari nilai $c$ agar fungsi $f$ kontinu di setiap bilangan real.

    Bagian pertama dari fungsi \textit{pointwise} ini adalah bentuk kuadratik, 
    dan yang kedua adalah bentuk linear. Keduanya kontinu di setiap bilangan real. 
    Sehingga titik yang mungkin menyebabkan diskontinuitas hanya di $x=2$.

    Tiga syarat untuk membuktikan $f$ kontinu di $x=2$ adalah
        \begin{enumerate}[label={\arabic*}.]
        \item $\lim_{x\to 2} f(x)$ ada.
        \item $f(2)$ ada.
        \item $\lim_{x\to 2} f(x) = f(2)$
        \end{enumerate}
    Karena fungsi $f$ diharapkan kontinu di setiap bilangan real, diasumsikan $f$ 
    juga memenuhi ketiga syarat diatas. Maka dari syarat pertama $\lim_{x\to 2} f(x)$ 
    ada. Jika limitnya ada maka limit kanan dan kirinya ada dan 
    \[
    \lim_{x\to 2^{-}} f(x) = \lim_{x\to 2^{+}} f(x)
    \]
    sehingga
    \begin{align*}
        \lim_{x\to 2^{-}} f(x) = \lim_{x\to 2^{+}} f(x)
        \iff &\lim_{x\to 2^{-}} c^{2}x^{2} = \lim_{x\to 2^{+}} (1-c)x\\
        \iff &c^{2}2^{2} = (1-c)2\\
        \iff &2c^{2} = 1-c\\
        \iff &2c^{2}+c-1 = 0\\
        \iff &(2c-1)(c+1) = 0
    \end{align*}
    Dari syarat pertama maka haruslah $c=-1$ atau $c=\frac{1}{2}$\\
    Misalkan $c=-1$ maka limit saat $x$ menuju 2 adalah 4 (Subtitusi ke limit kiri dan 
    kanannya). Akan dipastikan kedua syarat lainnya berlaku. Syarat kedua berlaku 
    dengan $f(2) = (1-(-1))2 = 4$. Syarat ketiga juga berlaku karena 
    $\lim_{x\to 2} f(x) = 4 = f(2)$.\\
    Misalkan $c=\frac{1}{2}$ maka limit saat $x$ menuju 2 adalah 1 (Subtitusi ke limit 
    kiri dan kanannya). Akan dipastikan kedua syarat lainnya berlaku. Syarat kedua berlaku 
    dengan
    \[
        f(2) = \brk*{1-\frac{1}{2}}2 = 1
    \] 
    dan syarat ketiga juga berlaku Karena
    \[
        \lim_{x\to 2} f(x) = 1 = f(2)
    \] 
    Sehingga kedua nilai $c=-1$ dan $c=\frac{1}{2}$ akan membuat fungsi $f$ 
    kontinu di setiap bilangan real.

    $\therefore$ Nilai $c$ agar fungsi $f$ kontinu di setiap bilangan real adalah $c=-1$ 
    dan $c=\frac{1}{2}$
    
    \end{enumerate}
\end{enumerate}

\begin{center}\line(1,0){300}\end{center}
