\begin{flushright}
    \textbf{\Large{Ujian Akhir Semester}}
    \subsection*{Tahun 2021}
    \addcontentsline{toc}{subsection}{UAS - 2021}
\end{flushright}


\vspace{0.5cm}\hrule height 2pt\vspace{0.5cm}


\begin{center}
\textbf{\large{MATERI}}
\begin{enumerate}[leftmargin=*, label={\arabic*}.]
\item Teorema Dasar Kalkulus
\item Aplikasi integral dalam mencari volume benda putar
\item Fungsi Invers
\item Fungsi Trigonometri dan Invers Trigonometri
\item Teknik-teknik Integrasi
\end{enumerate}
\end{center}


\vspace{0.2cm}\hrule height 1pt\vspace{0.5cm}


\begin{center}
\textbf{\large{SOAL}}
\end{center}
\begin{enumerate}[leftmargin=*, label={\arabic*}.]
\item Dengan menggunakan Teorema Dasar Kalkulus, tentukanlah $F'(x)$ jika diketahui:
\[
F(x) = \int_{-x^{2}}^{5x}\frac{s^{2}}{1+s^{2}}\,ds
\]
\item Sketsa dan carilah volume benda putar yang terbentuk, jika daerah yang 
diwarnai biru pada gambar di bawah diputar mengelilingi sumbu-$y$ dengan 
menggunakan:
    \begin{enumerate}[label={\alph*}.]
    \item Metode Cincin
    \item Metode Kulit Tabung
    \end{enumerate}

\begin{center}
\begin{tikzpicture}
    \realline;
    \draw (-5,1) --(-1.6,1);
    \draw (1.6,1) --(5,1);
    \draw (-1.6,1)--(-1.6,-.1) node[below=0.2em]{$-\sqrt{3}$};
    \draw (1.6,1)--(1.6,-.1) node[below=0.2em]{$\sqrt{3}$};
                
    \node at (3.3,.5) {$+$};
    \node at (-3.3,.5) {$+$};
    \node at (0,.5) {$-$};

    \point{white}{(-1.6,0)};
    \point{white}{(1.6,0)};
\end{tikzpicture}
\end{center}

\item Misalkan $\ds f(x)=\frac{a^{x}-1}{a^{x}+1}$ untuk $a$ tetap, $a>0$, $a\neq 1$.
    \begin{enumerate}[label={\alph*}.]
    \item Jelaskan mengapa $f$ mempunyai invers.
    \item Carilah $f^{-1}$.
    \end{enumerate}
\item Carilah $\ds \frac{dy}{dx}$ di titik $\ds P\brk*{\frac{1}{2},-\sqrt{2}}$ jika diketahui:
\[
y\arccos(xy) = \frac{-3\sqrt{2}}{4}\pi
\]
\item Hitunglah integral berikut:
    \begin{enumerate}[label={\alph*}.]
    \item $\ds \int\frac{2x^{3}+x^{2}+4}{\brk*{x^{2}+4}^{2}}\,dx$
    \item $\ds \int x^{3}e^{-3x}\,dx$
    \end{enumerate}
\end{enumerate}


\vspace{0.2cm}\hrule height 1pt\vspace{0.5cm}


\begin{center}
\textbf{\large{PEMBAHASAN}}
\end{center}
\begin{enumerate}[leftmargin=*, label={\arabic*}.]
\item Diberikan
\[
F(x) = \int_{-x^{2}}^{5x}\frac{s^{2}}{1+s^{2}}\,ds
\]
Akan dicari $F'(x)$ menggunakan Teorema Dasar Kalkulus.
\begin{align*}
    F'(x) &= \drv{x}{\int_{-x^{2}}^{5x}\frac{s^{2}}{1+s^{2}}\,ds}\\
    &= \drv{x}{\int_{-x^{2}}^{0}\frac{s^{2}}{1+s^{2}}\,ds
    +\int_{0}^{5x}\frac{s^{2}}{1+s^{2}}\,ds}\\
    &= \drv{x}{-\int_{0}^{-x^{2}}\frac{s^{2}}{1+s^{2}}\,ds
    +\int_{0}^{5x}\frac{s^{2}}{1+s^{2}}\,ds}\\
    &= \drv{x}{-\int_{0}^{-x^{2}}\frac{s^{2}}{1+s^{2}}\,ds}+
    \drv{x}{\int_{0}^{5x}\frac{s^{2}}{1+s^{2}}\,ds}\\
    &= \drv{(-x^{2})}{-\int_{0}^{-x^{2}}\frac{s^{2}}{1+s^{2}}\,ds}\frac{d(-x^{2})}{dx}+
    \drv{(5x)}{\int_{0}^{5x}\frac{s^{2}}{1+s^{2}}\,ds}\frac{d(5x)}{dx}\\
    &= \brk*{-\frac{x^{4}}{1+x^{2}}}(-2x)+\brk*{\frac{25x^{2}}{1+25x^{2}}}(5)\\
    &= \frac{2x^{5}}{1+x^{2}}+\frac{125x^{2}}{1+25x^{2}}\\
\end{align*}

$\therefore$ Diperoleh $\ds F'(x) =\frac{2x^{5}}{1+x^{2}}+\frac{125x^{2}}{1+25x^{2}}$.


\begin{center}\line(1,0){300}\end{center}


\item Diberikan daerah seperti gambar berikut

\begin{center}
\begin{tikzpicture}
    \realline;
    \draw (-5,1) --(-1.6,1);
    \draw (1.6,1) --(5,1);
    \draw (-1.6,1)--(-1.6,-.1) node[below=0.2em]{$-\sqrt{3}$};
    \draw (1.6,1)--(1.6,-.1) node[below=0.2em]{$\sqrt{3}$};
                
    \node at (3.3,.5) {$+$};
    \node at (-3.3,.5) {$+$};
    \node at (0,.5) {$-$};

    \point{white}{(-1.6,0)};
    \point{white}{(1.6,0)};
\end{tikzpicture}
\end{center}
    
Pertama akan disketsa volume benda putar.\\
Berikut hasil volume benda diputar terhadap sumbu-$y$.

\begin{center}
\begin{tikzpicture}
\begin{axis}[view={-30}{30},colormap/Spectral-9]
    \begin{scope}
        \def\rx{(5)}
        \def\fx{(4*x-x^2)}
        \addplot3[surf,shader=flat,
            samples=20,
            domain=0:2,y domain=0:2*pi,
            z buffer=sort]
            ({(((x-\rx) * cos(deg(y)))+\rx)}, {(((x-\rx) * sin(deg(y)))+\rx)}, {\fx});
    \end{scope}

    \begin{scope}
        \def\rx{(5)}
        \def\fx{(x^2)}
        \addplot3[surf,shader=flat,
            samples=20,
            domain=0:2,y domain=0:2*pi,
            opacity=0.25,
            z buffer=sort]
            ({(((x-\rx) * cos(deg(y)))+\rx)}, {(((x-\rx) * sin(deg(y)))+\rx)}, {\fx});
    \end{scope}
    
    \addplot3[black,shader=flat,
        samples=25,
        domain=0:4,
        z buffer=sort]
        (5,5,x);
    \node at (4,3.7,5) {$x=5$};
\end{axis}
\end{tikzpicture}       
\end{center}

Kedua akan dihitung volumenya
    \begin{enumerate}[label={\alph*}.]
    \item Menggunakan metode cincin, partisi tegak lurus sumbu-$y$.\\
    Panjang jari-jari luar cincin adalah jarak dari $x=1$ ke sumbu-$y$ yaitu $1$.\\
    Panjang jari-jari dalam cincin adalah jarak dari $x=\tan y$ ke sumbu-$y$ yaitu $\tan y$.\\
    Partisi dimulai dari $y=0$ sampai $y=\pi/4$ (saat $x=1$).

    Gunakan integral untuk menghitung volume.
    \begin{align*}
        V &= \int \text{Luas bagian luar} - \text{Luas bagian dalam}\\
        &=\int_{0}^{\pi/4} \pi(1)^2 - \pi(\tan y)^2\,dy\\
        &=\pi\int_{0}^{\pi/4} 1-\tan^{2} y\,dy\\
        &=\pi\int_{0}^{\pi/4} 1-(\sec^{2}y-1) y\,dy\\
        &=\pi\int_{0}^{\pi/4} 2-\sec^{2}y y\,dy\\
        &=\pi\eval{2y-\tan y}{0}{\pi/4}\\
        &=\pi\brk*{\brk*{2\frac{\pi}{4}-\tan\brk*{\frac{\pi}{4}}}-(0)}\\
        &=\pi\brk*{\frac{\pi}{2}-1}
    \end{align*}
    \item Menggunakan metode kulit tabung, partisi sejajar sumbu-$y$.\\
    Tinggi kulit tabung adalah jarak dari $x=\tan y$ ke $y=0$. yaitu $y=\arctan x$.\\
    Jari-jari kulit tabung adalah jarak $x=\tan y$ ke sumbu-$y$ yaitu dari $x=0$ 
    sampai $x=1$.

    Gunakan integral untuk menghitung volume
    \begin{align*}
        V &= \int \text{Luas Kulit Tabung}\\
        &=\int_{0}^{1} 2\pi(\text{jari-jari})(\text{tinggi})\,dx\\
        &=2\pi\int_{0}^{1} x\arctan x\,dx\\
        &\text{integral parsial dengan $u=\arctan x$ dan $dv=x\,dx$.}\\
        &=2\pi\brk*{\eval{\arctan x\brk*{\frac{1}{2}x^{2}}}{0}{1}
        -\int_{0}^{1}\brk*{\frac{1}{2}x^{2}}\frac{1}{1+x^{2}}\,dx}\\
        &=2\pi\brk*{\brk*{{\arctan (1)\brk*{\frac{1}{2}(1)^{2}}}-(0)}
        -\frac{1}{2}\int_{0}^{1}\frac{x^{2}}{1+x^{2}}\,dx}\\
        &=2\pi\frac{1}{2}\brk*{\arctan 1-\int_{0}^{1}\frac{x^{2}}{1+x^{2}}\,dx}\\
    \end{align*}
    \begin{align*}
        &\text{subtitusi $x=\tan u$ dan $dx=\sec^{2}\,du$}\\
        &=\pi\brk*{\arctan 1
        -\int_{\arctan 0}^{\arctan 1}\frac{\tan^{2}u}{1+\tan^{2}u}\sec^{2}u\,du}\\
        &=\pi\brk*{\arctan 1-\int_{0}^{\arctan 1}\frac{\tan^{2}u}{\sec^{2}u}\sec^{2}u\,du}\\
        &=\pi\brk*{\arctan 1-\int_{0}^{\arctan 1}\tan^{2}u\,du}\\
        &=\pi\brk*{\arctan 1-\int_{0}^{\arctan 1}\sec^{2}u-1\,du}\\
        &=\pi\brk*{\arctan 1-\eval{\tan u-u}{0}{\arctan 1}}\\
        &=\pi\brk*{\arctan 1-((\tan(\arctan 1)-\arctan 1)-(0))}\\
        &=\pi\brk*{2\arctan 1-1}\\
        &=\pi\brk*{2\frac{\pi}{4}-1}\\
        &=\pi\brk*{\frac{\pi}{2}-1}\\
    \end{align*}
    \end{enumerate}

$\therefore$ Telah disketsa benda putar dan diperoleh volumenya adalah 
$\ds \pi\brk*{\frac{\pi}{2}-1}$ satuan.


\begin{center}\line(1,0){300}\end{center}

    
    \item Misalkan $\ds f(x)=\frac{a^{x}-1}{a^{x}+1}$ untuk $a$ tetap, $a>0$, $a\neq 1$.
        \begin{enumerate}[label={\alph*}.]
        \item Akan dibuktikan $f$ mempunyai invers.
        
        Gunakan teorema berikut:
        \begin{theorem*}[Existence of Inverse Function]
            Jika $f$ monoton tegas pada domainnya, maka $f$ memiliki invers
        \end{theorem*}

        Akan dibuktikan $f$ monoton tegas, yaitu $f'(x) > 0$ atau $f'(x) < 0$ 
        untuk setiap $x$ di domainnya.
        \begin{align*}
            f'(x) &= \drv{x}{\frac{a^{x}-1}{a^{x}+1}}\\
            &=\frac{\drvL{x}{a^{x}-1}(a^{x}+1)-(a^{x}-1)\drvL{x}{a^{x}+1}}{(a^{x}+1)^{2}}\\
            &=\frac{a^{x}\ln a(a^{x}+1)-(a^{x}-1)a^{x}\ln a}{(a^{x}+1)^{2}}\\
            &=\frac{2a^{x}\ln a}{(a^{x}+1)^{2}}\\
        \end{align*}
        Diketahui $a > 0$ dan $a\neq 1$.\\
        Maka saat $0 < a < 1$, untuk setiap $x$ di domain
        \[
            f'(x)=\frac{(2a^{x})(\ln a)}{(a^{x}+1)^{2}} 
            = \frac{(\text{positif})(\text{negatif})}{(\text{positif})}
            =\text{(negatif)} < 0
        \]
        Sehingga $f$ monoton turun tegas.

        Lalu, saat $a > 1$, untuk setiap $x$ di domain
        \[
            f'(x)=\frac{(2a^{x})(\ln a)}{(a^{x}+1)^{2}} 
            = \frac{(\text{positif})(\text{positif})}{(\text{positif})}
            =\text{(positif)} > 0
        \]
        Sehingga $f$ monoton naik tegas.\\
        Terlihat untuk setiap nilai $a$ sesuai syarat, $f$ selalu monoton tegas. Dengan demikian, 
        sesuai teorema sebelumnya $f$ selalu memiliki invers.

        $\therefore$ Telah dibuktikan bahwa $f$ memiliki invers.


\begin{center}\line(1,0){150}\end{center}


        \item Akan dicari $f^{-1}$.
        Misalkan $f(x)=y$, maka
        \begin{align*}
            &y = \frac{a^{x}-1}{a^{x}+1}\\
            \iff & y(a^{x}+1) = a^{x}-1
            &\text{karena }a^{x}+1\neq 0\\
            \iff & ya^{x}+y = a^{x}-1\\
            \iff & y+1 = a^{x}-ya^{x}\\
            \iff & y+1 = a^{x}(1-y)\\
            \iff & a^{x} = \frac{1+y}{1-y}\\
            \iff & x = \log_{a}\brk*{\frac{1+y}{1-y}}\\
        \end{align*}
        Sehingga 
        \[
        f^{-1}(y) = x = \log_{a}\brk*{\frac{1+y}{1-y}}
        \]

        $\therefore$ Diperoleh $\ds f^{-1}(x)=\log_{a}\brk*{\frac{1+x}{1-x}}$
    
    \end{enumerate}

\begin{center}\line(1,0){300}\end{center}


\item Akan dicari $\ds \frac{dy}{dx}$ di titik $\ds P\brk*{\frac{1}{2},-\sqrt{2}}$ untuk:
\[
y\arccos(xy) = \frac{-3\sqrt{2}}{4}\pi
\]
turunkan secara implisit
\begin{align*}
    &\drv{x}{y\arccos(xy)} = \drv{x}{\frac{-3\sqrt{2}}{4}\pi}\\
    \iff &\brk*{\frac{dy}{dx}}\arccos(xy)+y\drv{x}{\arccos(xy)} = 0\\
    \iff &y'\arccos(xy)+y\drv{(xy)}{\arccos(xy)}\drv{x}{xy} = 0\\
    \iff &y'\arccos(xy)+y\brk*{-\frac{1}{\sqrt{1-x^{2}y^{2}}}}\brk*{y+x\frac{dy}{dx}} = 0\\
    \iff &y'\arccos(xy)+y\brk*{-\frac{1}{\sqrt{1-x^{2}y^{2}}}}\brk*{y+xy'} = 0\\
\end{align*}
Subtitusi $x=\frac{1}{2}$ dan $y=-\sqrt{2}$.
\begin{align*}
    &y'\arccos\brk*{\frac{1}{2}\brk*{-\sqrt{2}}}+\brk*{-\sqrt{2}}
    \brk*{-\frac{1}{\sqrt{1-(1/2)^{2}\brk*{-\sqrt{2}}^{2}}}}
    \brk*{\brk*{-\sqrt{2}}+\frac{1}{2}y'} = 0\\
    \iff &y'\arccos\brk*{\frac{-\sqrt{2}}{2}}+\sqrt{2}
    \brk*{\frac{1}{\sqrt{1-(1/4)(2)}}}
    \brk*{\brk*{-\sqrt{2}}+\frac{1}{2}y'} = 0\\
    \iff &y'\brk*{\frac{3\pi}{4}}+\sqrt{2}\brk*{\sqrt{2}}
    \brk*{\brk*{-\sqrt{2}}+\frac{1}{2}y'} = 0\\
    \iff &y'\brk*{\frac{3\pi}{4}}-2\sqrt{2}+y'=0\\
    \iff &3\pi y'+4y'=8\sqrt{2}\\
    \iff &y'=\frac{8\sqrt{2}}{3\pi+4}\\
\end{align*}

$\therefore$ Diperoleh $\ds \frac{dy}{dx}$ di titik $\ds P\brk*{\frac{1}{2},-\sqrt{2}}$ dari 
persamaan tersebut adalah $\ds \frac{8\sqrt{2}}{3\pi+4}$.


\begin{center}\line(1,0){300}\end{center}


\item Akan dicari integral berikut
    \begin{enumerate}[label={\alph*}.]
    \item $\ds \int\frac{2x^{3}+x^{2}+4}{\brk*{x^{2}+4}^{2}}\,dx$
    \[
        \int\frac{2x^{3}+x^{2}+4}{\brk*{x^{2}+4}^{2}}\,dx
        = \int\frac{2x^{3}}{\brk*{x^{2}+4}^{2}}\,dx
        +\int\frac{x^{2}+4}{\brk*{x^{2}+4}^{2}}\,dx
    \]
    Selesaikan masing-masing.
    \begin{align*}
        &\int \frac{2x^{3}}{(x^{2}+4)^2}\,dx\\
        =\,&\int \frac{x^{2}}{(x^{2}+4)^2}2x\,dx\\
        &\text{subtitusi $u=x^{2}+4$, $x^{2}=u-4$ dan $du = 2x\,dx$}\\
        =\,&\int \frac{u-4}{u^2}\,du\\
        =\,&\int \frac{1}{u}-\frac{4}{u^{2}}\,du\\
        =\,&\ln\abs{u}+\frac{4}{u}+C\\
        &\text{subtitusi balik $u=x^{2}+4$}\\
        =\,&\ln\abs*{x^{2}+4}+\frac{4}{x^{2}+4}+C\\
        =\,&\ln\brk*{x^{2}+4}+\frac{4}{x^{2}+4}+C
    \end{align*}
    Lalu 
    \begin{align*}
        &\int\frac{x^{2}+4}{\brk*{x^{2}+4}^{2}}\,dx\\
        =\,&\int\frac{1}{x^{2}+4}\,dx\\
        =\,&\int\frac{1}{x^{2}+2^{2}}\,dx\\
        =\,&\frac{1}{2}\arctan\brk*{\frac{x}{2}}+C
    \end{align*}
    Sehingga 
    \begin{align*}
        \int\frac{2x^{3}+x^{2}+4}{\brk*{x^{2}+4}^{2}}\,dx
        &= \int\frac{2x^{3}}{\brk*{x^{2}+4}^{2}}\,dx
        +\int\frac{x^{2}+4}{\brk*{x^{2}+4}^{2}}\,dx\\
        &=\ln\brk*{x^{2}+4}+\frac{4}{x^{2}+4}+\frac{1}{2}\arctan\brk*{\frac{x}{2}}+C
    \end{align*}

    $\therefore$ $\ds \int\frac{2x^{3}+x^{2}+4}{\brk*{x^{2}+4}^{2}}\,dx
    =\ln\brk*{x^{2}+4}+\frac{4}{x^{2}+4}+\frac{1}{2}\arctan\brk*{\frac{x}{2}}+C$


\begin{center}\line(1,0){150}\end{center}


    \item $\ds \int x^{3}e^{-3x}\,dx$
    
    Gunakan integral parsial secara berulang
    \begin{align*}
        &\int x^{3}e^{-3x}\,dx\\
        &\text{integral parsial dengan $u=x^{3}$ dan $dv=e^{-3x}\,dx$}\\
        =\,&x^{3}\brk*{-\frac{1}{3}e^{-3x}}-\int\brk*{-\frac{1}{3}e^{-3x}}3x^{2}\,dx\\
        =\,&-\frac{1}{3}x^{3}e^{-3x}+\int e^{-3x}x^{2}\,dx\\
        &\text{integral parsial dengan $u=x^{2}$ dan $dv=e^{-3x}\,dx$}\\
        =\,&-\frac{1}{3}x^{3}e^{-3x}+\brk*{x^{2}\brk*{-\frac{1}{3}e^{-3x}}
        -\int \brk*{-\frac{1}{3}e^{-3x}}2x\,dx}\\
        =\,&-\frac{1}{3}x^{3}e^{-3x}-\frac{1}{3}x^{2}e^{-3x}
        +\frac{2}{3}\int e^{-3x}x\,dx\\
        &\text{integral parsial dengan $u=x$ dan $dv=e^{-3x}\,dx$}\\
        =\,&-\frac{1}{3}x^{3}e^{-3x}-\frac{1}{3}x^{2}e^{-3x}
        +\frac{2}{3}\brk*{x\brk*{-\frac{1}{3}e^{-3x}}
        -\int\brk*{-\frac{1}{3}e^{-3x}}\,dx}\\
        =\,&-\frac{1}{3}x^{3}e^{-3x}-\frac{1}{3}x^{2}e^{-3x}
        +\frac{2}{3}\brk*{-\frac{1}{3}xe^{-3x}+\frac{1}{3}\int e^{-3x}\,dx}\\
        =\,&-\frac{1}{3}x^{3}e^{-3x}-\frac{1}{3}x^{2}e^{-3x}
        -\frac{2}{9}xe^{-3x}+\frac{2}{9}\int e^{-3x}\,dx\\
        =\,&-\frac{1}{3}x^{3}e^{-3x}-\frac{1}{3}x^{2}e^{-3x}
        -\frac{2}{9}xe^{-3x}+\frac{2}{9}\brk*{-\frac{1}{3} e^{-3x}+C}\\
        =\,&-\frac{1}{3}x^{3}e^{-3x}-\frac{1}{3}x^{2}e^{-3x}
        -\frac{2}{9}xe^{-3x}-\frac{2}{27} e^{-3x}+C\\
        =\,&-\frac{1}{3}e^{-3x}\brk*{x^{3}+x^{2}
        +\frac{2}{3}x+\frac{2}{9}}+C\\
    \end{align*}
    $\therefore$ $\ds \int x^{3}e^{-3x}\,dx 
    =-\frac{1}{3}e^{-3x}\brk*{x^{3}+x^{2}+\frac{2}{3}x+\frac{2}{9}}+C$
    \end{enumerate}
\end{enumerate}

\begin{center}\line(1,0){300}\end{center}
